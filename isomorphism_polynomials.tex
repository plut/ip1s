\documentclass{lms}%<<<1
%\usepackage[margin=30mm]{geometry}
%\usepackage[letterpaper,hmargin=1.4in,vmargin=1.4in]{geometry}
\usepackage{unicode}
\DeclareUnicodeCharacter{22F1}{\smash\ddots}
\DeclareUnicodeCharacter{22F0}{\smash\iddots}
\usepackage{amsmath,amssymb,mathrsfs}
\usepackage{bm}
\usepackage{mathdots}

\newnumbered{definition}{Definition}

% Theorems, equations, counters%<<<
\def\linkcounter#1#2{\edef\magic{\noexpand\let
  \expandafter\noexpand\csname c@#1\endcsname
  \expandafter\noexpand\csname c@#2\endcsname}\magic}
\def\newthm#1#2{\newtheorem{#1}{#2}[section]\linkcounter{#1}{equation}}
\newthm{prop}{Proposition}
\newthm{thm}{Theorem}
\newthm{lem}{Lemma}

\def\labelenumi{(\roman{enumi})} \let\itemref\ref
%>>>
% Misc. math stuff%<<<
\def\bigperp{\mathop{\vcenter{\hbox{\scalebox{2}{\ensuremath{\perp}}}}}%
  \displaylimits}
\let\fr\mathfrak
\let\ro\mathscr
\def\transpose{\,{}^{\mathrm{t}\!}}
\def\acco#1{\left\{#1\right\}}
\def\abs#1{\left|#1\right|}
\def\mat#1{\begin{pmatrix}#1\end{pmatrix}}
\def\card#1{\abs{#1}}
\DeclareMathOperator\Ker{Ker}
\DeclareMathOperator\End{End}
\def\F{\mathbb{F}}
\def\Id{\mathrm{Id}}
%>>>
\usepackage{color}
\def\commentaire#1{{\bfseries\textcolor{red}{#1}}}
%>>>1

\begin{document}
\title[Solving ``Isomorphism of Polynomials with Two Secrets'']%
{Solving the ``Isomorphism of Polynomials with Two Secrets'' Problem}%<<<1
%\author{Pierre-Alain Fouque \and Gilles Macario-Rat \and Jérôme Plût}
\author{Submission to ANTS 2014}
\maketitle

\commentaire{essai avec mise en ligne du PDF}
\begin{abstract}
In this paper, we study the Isomorphism of Polynomial (IP) problem with two
quadratic polynomials on $n$ variables over a finite field of odd
characteristic, \textit{i.e.} given two quadratic polynomials $(\bm{a},\bm{b})$ 
on $n$ variables, find two bijective linear maps $(s,t)$ such that 
$\bm{b}=t\circ \bm{a}\circ s$. We show a
polynomial-time algorithm to recover $s$ and $t$ in time complexity $\widetilde{O}(n^3)$.
This problem has been introduced in cryptography by Patarin back in 1996.
One special instance of this problem when $t$ is the identity is called
the isomorphism with one secret (IP1S) problem. At ASIACRYPT 2013, Macario-Rat, Pl\^ut and Gilbert
explained well behavior of generic solvers, such as Gr\"obner basis, to
solve the IP1S problem on random instances by describing a polynomial-time
algorithm for solving \textit{cyclic instances} using pencil of quadratic forms 
in \textit{all} finite fields. They show that this problem is
reducible to computing a square root of a matrix in the commuting space of 
some matrix, which is a commutative algebra for \textit{cyclic} matrices. Here, we solve all
instances of the problem by describing a new reduction algorithm for
pencils of bilinear forms in a non-commutative algebra. Finally, we show
that the general IP problem, a.k.a. IP2S problem is solvable in cubic
time.
\end{abstract}
% j'ai mis un tilde sur le big0 pour les termes en log(q) car il faut manipuler les scalaires de F_q

\section{Introduction}
The Isomorphism of Polynomial problem (IP) has been introduced in cryptography by Patarin in~\cite{DBLP:conf/eurocrypt/Patarin96} to construct an efficient authentication scheme 
as an alternative to the Graph Isomorphism Problem (GI) proposed by Goldreich, Micali and Wigderson~\cite{DBLP:journals/jacm/GoldreichMW91}. The IP problem was appealing since 
it seems more difficult than the Graph Isomorphism problem~\cite{DBLP:conf/eurocrypt/PatarinGC98}. 
A reduction between these
two problems has been formally proved by Agrawal and Saxena in~\cite{DBLP:conf/stacs/AgrawalS06}, 
using two polynomials. One of them is a quadratic polynomials and captures the adjacency 
matrices of the graphs and the other is cubic of the special form $\sum_{i} x_i^3$ over a 
finite field of odd characteristic. 
For the case of quadratic polynomials, the status of this problem is unclear despite recent 
intensive research in the cryptographic community since this case is the most interesting 
for practical schemes. There exists a claimed reduction between the quadratic IP1S problem and the GI
problem~\cite{DBLP:conf/eurocrypt/PatarinGC98}, but we realized that this proof is incomplete. 
Here, we are interested in the special case of two equations with homogeneous composants only 
since it seems to be the most difficult cases according~\cite{DBLP:conf/eurocrypt/Perret05,DBLP:conf/eurocrypt/FaugereP06,DBLP:conf/pkc/BouillaguetFFP11,DBLP:conf/eurocrypt/BouillaguetFV13}. 


\paragraph{Previous Work.}
The case with one equation is the equivalence problem between two quadratic forms which 
is a classical result of classification of quadratic forms~\cite{lidl1997finite}. The case with 
affine terms is easier since it is possible to obtain linear relations between the secret unknowns
for the IP1S problem and for the IP problem since $t$ is bijective~\cite{DBLP:conf/eurocrypt/PatarinGC98} 
and such information helps generic solvers, such as Gröbner basis algorithm, as it is shown in~\cite{DBLP:conf/eurocrypt/FaugereP06}.  
The case with more than two equations is easier since we can relinearized the 
systems~\cite{DBLP:conf/pkc/BouillaguetFFP11}. Consequently, the main difficult case is when the 
number of equations is two and the equations are homogeneous, \textit{i.e.} contain only quadratic terms. 

At Asiacrypt 2011, Bouillaguet, Fouque and Macario-Rat in~\cite{DBLP:conf/asiacrypt/BouillaguetFM11} 
use pencil of quadratic forms to recover the secret mappings $s$ and $t$ when $3$ equations are available
and one of the quadratic equations $\bm{a}$ comes from a special mapping $X\mapsto X^{q^\theta+1}$ 
over $\F_q$. In the case of the IP problem, this is optimal using an information theoretic arguments. 
At Asiacrypt 2013, Macario-Rat, Plût and Gilbert explained in~\cite{MPG2013}, how to
solve cyclic instances of the IP1S problem over finite fields of any characteristic. Cyclic instances are 
instances~$\bm{b} = (b_{∞}, b_{0})$ such that $b_{∞}$~is invertible and~$b_{∞}^{-1} b_{0}$ is cyclic.
Although the cyclic case is generic in the geometric sense (\emph{i.e.}
defined by the non-cancellation of some polynomial functions of the
coefficients of~$\bm{b}$), it is not the general case in a practical
sense.

%\commentaire{2. parler du travail de Jeremy + Faugere et Ludovic}
Finally, very recently Berthomieu, Faugère and Perret in~\cite{DBLP:journals/corr/BerthomieuFP13}
proposed an algorithm to solve the IP1S problem for any number of equations and claim that the 
time complexity is polynomial. However, some inconsistancies are present in this work and mismatches 
come from some parts of their algorithms work in zero characteristic fields while other parts are for 
finite fields. First of all, 
they solved the problem in an extension field of the base field. They turn the congruence problem, 
\textit{i.e.} given two systems of matrices $\{H_i\}_{1\leq i \leq m}$ and $\{H'_i\}_{1\leq i \leq m}$, 
find an invertible matrix $P$ such that $H'_i=\transpose{P}H_iP$ into a conjugaison problem, 
\textit{i.e.} given two systems of matrices $\{H_i\}_{1\leq i \leq m}$ and $\{H'_i\}_{1\leq i \leq m}$, 
find an invertible matrix $P$ such that $H'_i=P^{-1}H_iP$. Using a result proved in~\cite{dSP10} concerning 
the conjugaison problem, they
consider that it is equivalent to solve the problem in an extension field. However, there is cases where
the congruence problem has a solution in the extension but not in the base field. They assume that 
the two systems can always been transformed into two systems such that the first is the identity and 
this is the case if we can always find a square root, but this is not always possible. Then, their deterministic
algorithm relies on a result of~\cite{DBLP:conf/issac/ChistovIK97} and this work only deals with 
the field of real algebraic number or the field of algebraic number. Finally, the authors claim that their 
algorithm solves the decisional problem and not the computational problem.  

\paragraph*{Our contributions.}%<<<2
We first reduce the general IP1S problem to the \emph{regular case},
which is the case where $b_{∞}$~is invertible. This is the object of
section~\ref{S:IP1S-singular} of this document and uses the Kronecker
classification of pencils of quadratic forms.

We then prove that the regular case of the IP1S problem simplifies to a
reduction problem for some quadratic forms over a local algebra. As this
is a well-understood theory (in odd characteristic), we are able to give
a polynomial-time answer to all instances of IP1S in
section~\ref{S:IP1S-regular}.

The last section explains how we recover the second (``outer'') secret in
the two-secret problem.
\commentaire{ ? la partie singulière n'intervient pas et donc tout
secret~$h$ possible sur la partie régulière est bon ?}

\section{Mathematical Background}%<<<2
Let~$V$ be a $n$-dimensional vector space over the field~$k$. We study
the IP1S and IP2S problems for \emph{quadratic forms} on~$V$, which are
homogeneous polynomials of degree~$2$ in some coordinates on~$V$. To a
quadratic form~$q$, one may associate the \emph{polar form}~$b$ defined
by
\begin{equation*}\label{eq:polar}
b(x,y) = q(x+y) - q(x) - q(y);
\end{equation*}
this is a symmetric bilinear forms, and it satisfies the \emph{polarity
identity}
\begin{equation*}\label{eq:polarity}
b(x,x) = 2q(x).
\end{equation*}
If~$2 ≠ 0$ in~$k$, then the polarity identity is a bijection between
quadratic forms and bilinear forms. Therefore, instead of quadratic
forms, we shall study directly bilinear forms.

We note that, if~$2 = 0$ in~$k$, the situation is much more complicated;
the polarity identity is no longer a bijection, but polar forms are
instead alternate bilinear forms. This means that their classification is
very different from the odd-characteristic case, and relies on symplectic
groups and Artin-Schreier type equations, \emph{i.e.} of the
type~$x^2+x+C = 0$.


\begin{definition}%\textbf{[Quadratic IP1S]}% \\
The IP1S problem is the following. 
Given two $m$-tuples $\bm{a} = (a_1, \ldots, a_{m})$ and {$\bm{b} = (b_1, \ldots, b_{m})$}  of quadratic homogeneous forms in $n$ variables over $k$, find a non-singular linear mapping $s \in \mathrm{GL}_n(k)$ (if any) such that $\bm{b} = \bm{a} \circ s$, i.e. $a_i = b_i \circ s$ for $i= 1, \ldots, m$.  
\end{definition}

%>>>1
\section{IP1S in the possibly singular case}%<<<1
\label{S:IP1S-singular}
\subsection{Regular and singular pencils}%<<<2
Throughout this document, $k$~is a field such that~$2 ∈ k^{×}$. Let~$V$
be a vector space. An \emph{(affine) pencil of symmetric bilinear forms}
over~$V$, or a \emph{symmetric pencil} in short, is the data of a pair of
symmetric bilinear forms~$\bm{b} = (b_{∞}, b_{0})$ over~$V$. We write
this pencil in affine form as~$b_{λ} = -λ b_{∞} + b_{0}$, and in
projective form as~$b_{λ:μ} = -λ b_{∞} + μ b_{0}$, where $b_{0}$
and~$b_{∞}$ are symmetric bilinear forms.

The \emph{characteristic polynomial} of the symmetric pencil~$(b_{λ})$ is
either the polynomial~$f(λ) = \det (λ b_{∞} + b_{0})$, or its homogeneous
form~$f(λ: μ) = \det (λ b_{∞} + μ b_{0})$. If $\dim_{k} V = n$, then
$f(λ: μ)$~is homogeneous of degree~$n$. The pencil~$(b_{λ})$ is called
\emph{regular} if the characteristic polynomial is not zero, and
\emph{singular} otherwise. We solve the isomorphism problem for regular
pencils in section~\ref{S:IP1S-regular} below.

We reduce to the regular case by proving that the singular part of a
symmetric pencil is reducible to the canonical form of Kronecker. This
form is described in~\cite[XII(56)]{Gantmacher2}; however, the proof
given there only applies to pencils over~$ℂ$, as its uses the computation
of square roots of matrices via interpolation on the spectrum. We give
here a proof that is true over any field~$k$ such that~$2 ≠ 0$ and which
is algorithmic.

\subsection{Reduction of (possibly singular) pencils to the Kronecker form} %<<<2
% Définition: indice minimal%<<<
The pencil~$(b_{λ})$ defines a symmetric bilinear form on the
module~$V_{∞} = V ⊗_{k} k[λ]$; if $(b_{λ})$~is singular, then this form
has a non-trivial kernel~$W$. Elements of~$W$ are called \emph{relations}
of~$(b_{λ})$. An element~$x = x_0 + λ x_1 + … + λ^d x_d$ is a
relation iff
\begin{equation}\label{eq:relation}
b_0 x_0 = 0, \quad
b_0 x_1 = b_{∞} x_0, \quad …
b_0 x_d = b_{∞} x_{d-1}, \quad
b_{∞} x_{d} = 0.
\end{equation}
A \emph{minimal relation} for~$(b_{λ})$ is one with minimal degree~$d$;
this degree is the \emph{minimal index} of~$(b_{λ})$. A basis
of~$W$ adapted to the filtration of~$V_{∞}$ by the degree of polynomials,
we see that $W$~as a basis~$(w_1,…,w_r)$ such that, if $d_i$~is the
degree of the relation~$w_i$, then $(w_i,…,w_r)$~generate no relation of
degree~$< d_i$. The degree~$d_i$ are called the \emph{minimal indices} of
the pencil~$(b_{λ})$.
%>>>
\begin{prop}\label{prop:minimal-indep}%<<<
Let~$e = ∑ λ^i e_i$ be a minimal relation for~$(b_{λ})$. Then
\begin{enumerate}
\item The~$d+1$ vectors~$e_0, …, e_d$ are $k$-linearly independent.
\item The~$d$ linear forms~$b_{0} e_1, …, b_{0} e_d$ are $k$-linearly
independent.
\end{enumerate}
\end{prop}

\begin{proof}
We first prove~(ii). Assume that there exists a non-trivial linear
relation~$α_1 e_0 + … + α_d e_{i}$ satisfies the relations
\begin{equation}\label{eq:relation-e'}
\begin{split}
b_0 e'_0 &= α_d b_0 e_0 = 0, \\
b_0 e'_i &= α_{d-i+1} b_0 e_1 + … + α_m b_0 e_{i-1} = b_{∞} e'_{i-1}, \\
b_{∞} e'_{d-1} &= b_{∞} (α_1 e_{0} + … + α_{d} e_{d-1}) \\
 &= b_{0} (α_1 e_{1} + … + α_{d} e_{d}) \\
 &= 0.
\end{split}
\end{equation}
This means that~$(e'_0,…,e'_{d-1})$ is a relation of degree~$≤ d-1$
for~$(b_{λ})$, which contradicts the minimality of~$e$.

To prove~(i), let~$α_0 e_0 + … + α_d e_d = 0$ be a non-trivial linear
relation. Then since $α_1 b_0(e_1) + … + α_d b_0(e_d) = 0$, by~(ii) we
must have~$α_1 = … = α_d = 0$, which in turn implies~$e_0 = 0$. However,
in this case we see that~$(e_1,…,e_d)$ is a relation of degree~$≤ d-1$
for~$(b_{λ})$.
\end{proof}

This means that there exists a basis of~$V$ in which the pencil~$(b_{λ})$
has the matrix
\begin{equation}\label{eq:minimal-matrix}
B_{0} = \mat{0 & \transpose{C_0}\\ C_0 & B'_{0}}, \quad
B_{∞} = \mat{0 & \transpose{S} \transpose{C_0}\\ C_0S & B'_{∞}}.
\end{equation}
where $0$~is a square block of size~$(d+1)$, $C_0$~is a matrix of size~$n'
× (d+1)$ with null first column and rank~$d$, $S$~is the left-shift
matrix of size~$d+1$, and $B'_{∞}, B'_0$~are symmetric matrices of
size~$n'$.

%>>>

From~\eqref{eq:minimal-matrix}, using a change of basis, we deduce
that there exists a basis in which the pencil~$(b_{λ})$ has the matrix
\begin{equation}
B_0 = \mat{0 & \transpose{C_0} & 0 \\ C_0 & 0 & 0 \\ 0 & 0 & B'_0},\qquad
B_{∞} = \mat{0 & \transpose{S} \transpose{C_0} & 0 \\
  C_0 S_0 & 0 & 0 \\ 0 & 0 & B'_0}.
\end{equation}
where $C_0 = \mat{0 & 1 & & \\⋮& &⋱&\\0&&&1\\}_{d×(d+1)}$ and $S =
\mat{0&&&0\\ 1&⋱&&\\&⋱&⋱&\\0&&1&0}_{(d+1)}$~is the left-shift matrix.

\commentaire{Un lecteur attentif verra que l'argument ci-dessus est
bidon. Je ne sais pas prouver exactement le résultat que je veux (mais
expérimentalement ça marche, il n'y a pas d'arnaque avec des non-carrés
par exemple). Je vais essayer d'avoir une preuve avant la deadline.}

\begin{thm}[(Kronecker form)]%<<<
Let~$(b_{λ})$ be a symmetric pencil on~$V$. There exists a basis of~$V$
in which the pencil has a block-diagonal matrix with diagonal
blocks~$(K_{d_1}, …, K_{d_r}, B')$, where $K_d$~is the square matrix of
size~$2d+1$ defined by
\begin{equation}
\def\arraystretch{.9}
K_d = \mat{0 & \transpose{K'_d}\\K'_d & 0}, \quad
% K'_d = \mat{ λ& &0\\ 1&⋱& \\ &⋱&λ\\ 0& &1}_{d×(d+1)},
K'_d = \mat{λ&1& &0\\ &⋱&⋱& \\0& &λ&1\\}_{d×(d+1)},
% L_{d} = \mat{
%  & & & &λ& &0\\
%  &0& & &1&⋱& \\
%  & & & & &⋱&λ\\
%  & & & &0& &1\\
% λ&1& &0& & & \\
%  &⋱&⋱& & &0& \\
% 0& &λ&1& & & \\
% },
\end{equation}
the integers~$d_1 ≤ … ≤ d_r$ are the minimal indices of~$(b_{λ})$, and
$B'$~is the matrix of a regular pencil.
\end{thm}
Note that in particular, the matrix~$K_{0}$ is the zero matrix of
size~$1×1$. The appearances of~$K_0$ in the Kronecker form correspond to
the constant vectors in the kernel of~$B_{λ}$.
%>>>
\section{IP1S for regular pencils}%<<<1
\label{S:IP1S-regular}

We give here an algorithm for solving the IP1S problem in the case of two
regular pencils. Assume that $\bm{b} = (b_{λ})$~is regular, and let~$f ≠ 0$
be its characteristic polynomial. Then, for any~$λ$ such that~$f(λ) ≠ 0$,
the bilinear form~$b_{λ}$ is regular.

\subsection{Localisation of the IP1S problem}%<<<2

\begin{lem}\label{lem:decomp-bezout}%<<<
Let~$\bm{b}$ be a regular symmetric pencil on the vector space~$V$. Let~$f(λ:
μ)$~be the homogeneous characteristic polynomial of~$S$, and let~$f = ∏
g_i$ be a factorisation of~$f$ in mutually coprime factors.

Then there exists a unique decomposition~$V = ⨁ V_i$ such that the
spaces~$V_i$ are pairwise orthogonal for all forms of~$\bm{b}$ and the
restriction~$S|_{V_i}$ has characteristic polynomial~$g_i$.
\end{lem}

\begin{proof}
Let~$V_{∞} = \Ker b_{∞}$ and $V'$ be the orthogonal of~$b_{∞}$ relatively
to the bilinear form~$b_{0}$. Then the decomposition~$V = V' ⊕ V_{∞}$ is
orthogonal for all forms~$b_{λ}$; replacing~$V$ by~$V'$, we may assume
that $b_{∞}$~is a regular bilinear form. This implies that $b_0$~has an
adjoint endomorphism~$m = b_{∞}^{-1} b_0$ such that~$b_0(x,y) = b_{∞}(x,
ay) = b_{∞}(ax, y)$; in particular, all elements of the algebra~$k[m]$
are self-adjoint with respect to~$b_{∞}$.

Let~$f(λ) = f(λ: 1)$ be the affine characteristic polynomial. It is
enough to prove the result for the decomposition~$f = gh$ where $g, h$
are mutually prime. Let~$u, v$ be polynomials such that~$ug + vh = 1$ and
$x, y ∈ V$ such that~$g(m)(x) = 0$ and~$h(m)(y) = 0$; we may then write
\begin{equation}\label{eq:bezout}
\begin{split}
b_{∞} (x, y) & = b_{∞} (x,\: u(m) g(m) y + v(m) h(m) y ) \\
&= b_{∞} (u(m) g(m) x, y) \,+\, b_{∞} (x, v(m) h(m) y) \\
&= 0.
\end{split}
\end{equation}
Since~$y' = ay$ also verifies~$h(m)(y') = 0$, equation~\eqref{eq:bezout}
also proves that~$b_{0}(x,y) = b_{∞}(x,y') = 0$, and hence~$x, y$ are
orthogonal for all forms~$b_{λ}$.
\end{proof}%>>>

The decomposition of~$V$ obtained by applying
Lemma~\ref{lem:decomp-bezout} to the full factorisation of~$f$
over~$k[x]$ is the \emph{primary decomposition} of the pencil~$(b_{λ})$.
The restriction of the pencil to each summand~$V_i$ has as its
characteristic polynomial a power of a prime polynomial; such a pencil is
called \emph{local}.

If two regular pencils~$\bm{b}, \bm{b}'$ are isomorphic (in the IP1S sense), then
they have the same characteristic polynomial, and computing an
isomorphism between~$\bm{b}$ and~$\bm{b}'$ is the same as computing it on each
factor of the primary decomposition. Therefore, in what follows, we shall
assume that both pencils are local.

\medskip

Note that when $k = \F_q$~is a finite field, it may happen that $λ^q μ -
λ μ^q$ divides~$f(λ:μ) ≠ 0$, so that~$f(λ) = 0$ for all~$λ ∈ ℙ^1(k)$. In
this case, although $(b_{λ})$~is a regular pencil, all forms~$b_{λ}$ are
degenerate. However, the decomposition given by
Lemma~\ref{lem:decomp-bezout} still applies, and all the local pencils
given by this decomposition contain at least one non-degenerate form,
namely~$b_{0}$ on~$V_{∞}$ and~$b_{∞}$ in all other cases.
Swapping~$b_{∞}$ and~$b_{0}$ when required, we may assume that all local
pencils are \emph{finite}, \emph{i.e.} that $b_{∞}$~is a regular bilinear
form.

\bigskip

Let~$b_{λ} = λ b_{∞} + b_0$ be a finite pencil. We may use the
characteristic endomorphism~$m = b_{∞}^{-1}b_0$ to write~$\bm{b}$ in the form
\begin{equation}\label{eq:adjoint}
b_{λ} = b_{∞}\;(λ + m).
\end{equation}
The image of~$\bm{b}$ by a linear change of variables~$s$ is then
\begin{equation}\label{eq:adjoint-change}
\transpose{s} · b_{λ} · s = 
  \transpose{s} · b_{∞} · s\; ( λ + s^{-1} · m · s).
\end{equation}
Let~$b'_{λ} = b'_{∞} (λ + m')$ be a pencil isomorphic to~$\bm{b}$; then there
exists a change of variables~$t$ such that~$t^{-1} · m' · t = m$, and the
pencil~$\transpose{t} · b'_{λ} · t$ is of the form $b'_{∞} (λ + m)$, so
that we may assume that~$m' = m$. Computing the isomorphism
between~$b_{λ}$ and~$b'_{λ}$ then amounts to computing~$s$ such that
\begin{equation}
\transpose{s} · b_{λ} · s = b'_{λ}, \qquad s · m = m · s.
\end{equation}

We define the \emph{symmetrizing
space}~$\ro S(m)$ and the \emph{commutant}~$\ro C(m)$ as
\begin{equation}\begin{split}
\ro S(m) &= \acco{\text{$b$ symmetric bilinear such that $bm$~is symmetric} },\\
\ro C(m) &= \acco{\text{$a ∈ \End V$ such that~$am = ma$}}.
\end{split}\end{equation}
The invertible elements of~$\ro C(m)$ form the \emph{commutant
group}~$\ro C(m)^{×}$.

\begin{prop}\label{prop:structure-sym}
Let~$m$ be an endomorphism of~$V$.
\begin{enumerate}
\item \label{it:sym-inv} The set~$\ro S(m)$ contains a regular bilinear
form.
\item \label{it:sym-comm} For any regular bilinear form~$t ∈ \ro S(m)$
and any endomorphism~$a$ of~$V$, the bilinear form~$ta$ belongs to~$\ro
S(m)$ if and only if $a$~is self-adjoint with respect to~$t$ and $a ∈ \ro
C(m)$.
\item Any finite pencil with characteristic endomorphism~$m$ is of the
form $b_{λ} = ta (λ+m)$ where~$a ∈ \ro C(m)^{×}$.
\item Let~$b_{λ} = ta(λ+m)$ be a finite pencil and~$s ∈ \ro C(m)$.
Let~$s^{⋆} = t^{-1} · \transpose{s} · t$ be the adjoint of~$s$
relatively to the bilinear form~$t$. Then
\begin{equation}
\transpose{s} · b_{λ} · s = t\: (s^{⋆} a s)\: (λ + m).
\end{equation}
% \item Let~$a ∈ \ro C(m)$. Then $\transpose{(t·a)} = t·a^{⋆}$. In
% particular, $t·a$~is symmetric if, and only if, $a = a^{⋆}$.
\end{enumerate}
\end{prop}

\begin{proof}
Point~\itemref{it:sym-inv} is explicitly proven in Prop.~\ref{prop:big-T}
below. Assuming that $a$~is self-adjoint with respect to~$t$,
point~\itemref{it:sym-comm} follows from
\begin{equation}
\transpose{(tam)} \;=\; \transpose{m}\, \transpose{(ta)}
 \;=\; \transpose{m}\,ta = tma;
\end{equation}
since $t$~is regular, it is cancellable in the resulting equation~$tma
= tam$.
\end{proof}

\begin{prop}\label{prop:IP1S-congruence}
The congruence problem for finite symmetric pencils is equivalent to the
following: given an endomorphism~$m$ of~$V$ and two invertible
self-adjoint matrices~$a, a' ∈ \ro C(m)$, compute a matrix~$x ∈ \ro
C(m)^{×}$ such that~$x^{⋆} a x = a'$.
\end{prop}

In the particular case where $m$~is cyclic, the commutant~$\ro C(m)$ is
the (commutative) polynomial algebra~$k[m]$. In this case, solving the
IP1S problem is straightforward~\cite{MPG2013}. The proof given here
specializes in the cyclic case to the proof of~\cite{MPG2013}. Namely, if
$m$~is cyclic, then Prop.~\ref{prop:IP1S-congruence} amounts to
equivalence of 1-dimensional quadratic forms over~$k[m]$, which is simply
a square root computation.

\subsection{The local IP1S problem in matrix form}%<<<2

Assume that $b_{λ}$~is a finite local pencil with characteristic
endomorphism~$m$ and characteristic polynomial~$p^d$, where $p$~is an
irreducible polynomial of degree~$e$. Let~$M_p$ be the companion matrix
of~$p$; then $k[M_p]$~is isomorphic to the extension field~$K =
k[μ]/(p(μ))$. For any matrix~$A = (a_{i,j})$ of size~$u × v$ with
coefficients in~$K$, let~$A^{♭}$ be the (``flattened'') matrix of
size~$eu × ev$ with blocks of size~$e × e$ given by~$a_{i,j}(M_p)$. We
have~$1^{♭} = 1$ and~$(AB)^{♭} = A^{♭} B^{♭}$ whenever the product
exists. We write~$A ↦ A^{♯}$ for the inverse map where it is defined.

For any integer~$u$, let~$H_u$ be the companion matrix of the
polynomial~$x^u$. Then there exists a basis of~$V$ in which the
matrix~$M$ of~$m$ is block-diagonal, with diagonal blocks~$M_{n_i} =
(H_{n_i} + μ)^{♭}$, for integers~$n_1 ≥ … ≥ n_r$.


\begin{lem}\label{lem:commute-jordan}%<<<
For all integers~$u, v$, define a matrix~$J_{u,v}$ of size~$u × v$ as
\begin{equation}
\def\arraystretch{.9}
J_{u,v} = \mat{1& &0\\&⋱&\\0&&1\\0&⋯&0\\⋮&&⋮\\0&⋯&0} \text{ if~$u ≥
v$,}\quad
J_{u,v} = \mat{0&⋯&0&1&&0\\⋮&&⋮&&⋱&\\0&⋯&0&0&&1} \text{ if $u ≤ v$.}
\end{equation}
\begin{enumerate}
\item For all integers~$u, v$, $H_{u} J_{u,v} = J_{u,v} H_{v}$.
\item The space of all matrices~$A$ of size~$u × v$ such that~$H_u A = A
H_v$ is exactly $k[H_u] J_{u,v}$.
\item \label{it:Juw} $J_{u,v} J_{v,w} = H_v^{d} J_{v,w}^{}$ where $d$~is
the distance between~$v$ and the interval~$[u,w]$. In particular,
$J_{u,v} J_{v,u} = H_{u}^{\abs{u-v}}$.
\end{enumerate}
\end{lem}
%>>>
\begin{prop}\label{prop:structure-commutant}%<<<
Let~$M$ be the block-diagonal matrix with diagonal blocks~$(H_{n_i} +
μ)^{♭}$. The commuting space of~$M$ is the space of all matrices~$A^{♭}$,
where $A$~is a block matrix~$A = (A_{i,j})$, with $A_{i,j} ∈
 k[H_{n_i}] J_{n_{i}, n_{j}} = J_{n_i, n_j} k[H_{n_j}]$.
\end{prop}
%>>>
According to Prop.~\ref{prop:structure-commutant}, we may replace all
elements~$A$ of~$\ro C(M)$ by their images~$A^{♯}$ in~$\ro C(M^{♯})$. We
may therefore assume that~$e = 1$, which means that~$K = k$. In this
case, as~$\ro C(M) = \ro C(M-μ)$, we may further assume that~$μ = 0$.
This means that  $M$~is the block-diagonal matrix with diagonal
blocks~$H_{n_i}$ for integers~$n_1 ≥ … ≥ n_r$.

Let~$A ∈ \ro C(M)$. Each entry~$A_{i,j}$ may be written as a polynomial
\begin{equation}
A_{i,j} = a_{i,j} (H_{n_i}) J_{n_i,n_j} = J_{n_i,n_j} a_{i,j} (H_{n_j})
\end{equation}
where $a_{i,j}(H) ∈ k[H]/H^{n_j}$.

We simplify the notation and write~$A = (a_{i,j})$ where~$a_{i,j} ∈
k[H]$. We note however that elements of~$\ro C(M)$ do not multiply as
matrices with coefficients in~$k[H]$, due to the relations
of Lemma~\ref{lem:commute-jordan}. An easy way to perform the computations is
given in Prop.~\ref{prop:phantom} below.

\begin{prop}\label{prop:phantom}%<<<
Let~$R = k[H]/H^{n_1}$. For all~$i, j$, let~$e_{i,j} = \max (0, n_i -
n_j)$. For any matrix~$A = (A_{i,j}) ∈ \ro C(M)$, where $A_{i,j} =
a_{i,j}(H_{n_i}) J_{n_i, n_j}$, define
\begin{equation}
ψ(A) = (H^{e_{i,j}}\; a_{i,j}(H)) \; ∈ \; R^{r×r}.
\end{equation}
Then $ψ$~is a $k$-algebra morphism from~$\ro C(M)$ to~$R^{r×r}$.
\end{prop}

\begin{proof}
The matrices of~$\ro C(M)$ multiply according to the relations of
Lemma~\ref{lem:commute-jordan}. We only need to prove that the
integers~$e_{i,j}$ are an integral of the exponents~$\mathrm{distance}
(n_k, [n_i, n_j])$ of Lemma~\ref{lem:commute-jordan}:
\begin{equation}\label{eq:integral}
e_{i,k} - e_{i,j} + e_{k,j} = \mathrm{distance} (n_k, [n_i, n_j]).
\end{equation}
This can be verified by checking for all orderings of the
triple~$(i,j,k)$.
\end{proof}%>>>

\paragraph{The adjunction involution.}
We now describe the adjunction involution~$A ↦ A^{⋆}$ of the commuting
space~$\ro C(M)$.

\begin{prop}\label{prop:big-T}%<<<
For each integer~$u$, write~$T_u$ for the anti-identity matrix of
size~$u$.
\begin{enumerate}
\item $T_u$~is invertible and both~$T_u$ and~$T_u H_u$ are symmetric.
\item $\transpose{J_{u,v}} T_u = T_v J_{v,u}$.
\item Let~$T$ be the block-diagonal matrix with diagonal
blocks~$T_{n_i}$. Then $T ∈ \ro S(M)^{×}$.
\end{enumerate}
\end{prop}
%>>>
\begin{prop}\label{prop:adjoint}%<<<
Let~$A = (a_{i,j}(H)) ∈ \ro C(M)$. Then \[ A^{⋆} = (a_{ji}(H)). \]
In particular, $TA$~is symmetric if, and only if, $a_{i,j} = a_{ji}$
in~$k[H]/H^{\min (n_i, n_j)}$.
\end{prop}
%>>>

As a corollary of Prop.~\ref{prop:phantom} and~\ref{prop:adjoint}, we get
the following. Let~$D$ be the diagonal matrix~$(H^{n_1-n_i}) ∈ R^{r×r}$.
For any~$A = (a_{i,j}∈ \ro C(M)$, let
\begin{equation}
φ(A) = D · ψ(A) = (H^{\max (n_1-n_i,n_1-n_j)}a_{i,j}) ∈ R^{r×r}.
\end{equation}
We then have~$φ(A^{⋆}) = \transpose{φ(A)}$. This implies that, for
all~$A, X ∈ \ro C(M)$:
\begin{equation}\label{eq:phi-psi}
φ(X^{⋆}\,A\,X) = \transpose{ψ(X)}\, φ(A)\, ψ(X).
\end{equation}

\paragraph{Structure of the commutant group.}%<<<

We show here how to compute in the commutant group~$\ro C(M)^{×}$. The
main result can be stated in two equivalent ways, as a formula to compute
the determinant of elements of~$\ro C(M)^{×}$, or as a Gaussian
elimination algorithm in the commutant group. For~$A ∈ \ro C(M)$, we
define the \emph{big block} of index~$(u,v)$ of~$A$ as the
sub-matrix~$(a_{i,j})$, where $i,j$~run over the range where~$n_i = u$
and~$n_j = v$. Prop.~\ref{prop:det-bigblock} states that $A$~is
invertible if and only if all its diagonal big blocks are invertible as
matrices with coefficients in~$k[H]/H^{n_i}$.

\begin{lem}\label{lem:det-1block}%<<<
Let~$s$ be such that~$n_1 = … = n_s > n_{s+1}$.
Define~$n'_i = n_i - 1$ if~$i ≤ s$, and~$n'_i = n_i$ if~$i > s$.

For any matrix~$A ∈ \ro C(M)$ given by the block decomposition~$A =
(a_{i,j}(H_{n_i}) J_{n_i,n_j})$, define~$A'$ as the square matrix of
size~$n - s$ given by the blocks~$(a_{i,j}(H_{n'_i}) J_{n'_i,
n'_j})$.
Then \[ \det A \;=\; \det (a_{i,j}(0))_{1 ≤ i,j ≤ s} · \det A'. \]
\end{lem}

\begin{proof}
Let~$σ$ the unique permutation of~$[1,n]$ such that~$σ(n_i) = i$ for~$i =
1, …, s$, and~$σ$ is increasing on all other indices. Then the matrix
$A^{σ}$ deduced from~$A$ by applying~$σ$ both on the lines and the
columns of~$A$ is lower block-triangular, with two diagonal blocks
respectively equal to the matrix~$(a_{i,j}(0))_{i,j ≤ s}$ and to~$A'$.
Since~$\det A = \det A^{σ}$, this proves the lemma.
\end{proof}
%>>>
\begin{prop}\label{prop:det-bigblock}%<<<
Let~$A = (a_{i,j}(H)) ∈ \ro C(M)$. Then
\[ \det A \;=\; ∏_{d ≥ 1} \det (a_{i,j}(0) | n_{i} = n_j = d)^{d} \]
where $a_{i,j}(0)$~is the image modulo~$H$ of~$a_{i,j} ∈ k[H]$.
\end{prop}

\begin{proof}
By induction on~$n_1$, with Lemma~\ref{lem:det-1block} providing the
induction step. The base case~$n_1 = 1$ corresponds to~$k[H] = k$ and
therefore~$A = (a_{i,j}(0))$.
\end{proof}
%>>>
\begin{prop}\label{prop:structure-gl}%<<<
The commutant group~$\ro C(M)^{×}$ is
generated by the following matrices:
\begin{enumerate}
\item big-block-diagonal matrices, i.e. matrices whose only non-zero big
blocks are those on the diagonal;
\item small-block transvection matrices.
\end{enumerate}
\end{prop}

\begin{proof}
By Prop.~\ref{prop:det-bigblock}, an invertible matrix~$A ∈ \ro C(M)$ has
all its diagonal big blocks invertible. Therefore we may apply the
Gaussian elimination algorithm to factor $A$~as a product~$A = LU$, where
$L$~is lower triangular with diagonal elements~$1$, and $U$~is upper
triangular.
\end{proof}%>>>
%>>>

\subsection{Classification of local symmetric pencils}%<<<2

We shall need the following classical result about bilinear forms over
local algebras.

\begin{prop}\label{prop:local-diag}%<<<
Let~$R$ be a local ring with residue field~$k$ and let~$\acco{δ_i}$ be a
set of representatives for~$k^{×}$ modulo squares. Then any regular
symmetric matrix with entries in~$R$ is congruent over~$R$ to a
diagonal matrix~$(1, …, 1, δ_i)$ for some~$i$.
\end{prop}

\begin{proof}
The Gram orthogonalization algorithm works; cf.
\cite[I(3.4)]{milnorhusemoller} or \cite[92:1]{omeara}.
\end{proof}

In the present case, since $k$~is a finite field, we have $k^{×} /
(k^{×})^2 = \acco {1, δ}$ for some element~$δ$.
%>>>
\begin{prop}\label{prop:bb-diag}%<<<
Let~$A$ be an invertible, self-adjoint element of~$\ro C(M)$. Then there
exists $X ∈ \ro C(M)^{×}$ such that $X^{*} A X$~is big-block-diagonal.
\end{prop}

\begin{proof}
We prove this by induction on the number of big blocks of~$A$, using Gram
orthogonalization on the big blocks of~$A$. Let~$B$ be the first diagonal
big block of~$A$ and~$d$ be the size of the first big block of~$a$.
We may then write $A$~as a block matrix
\begin{equation}
A = \mat{B & C \\ \transpose{C} & A'}, B ∈ R^{d×d}.
\end{equation}
Since $A$~is invertible, all its diagonal big blocks are invertible; in
particular the matrices~$B$ and $A'$~are invertible. By the induction
hypothesis, there exists~$X'$ such that~$Δ = (X')^{*} A' X'$ is
big-block diagonal.

We then define
\begin{equation}
X = \mat{1 & -B^{-1} C\\0 & X'}\quad\text{and see that}\quad
X^{⋆} A X = \mat{B & 0\\0 & Δ}.
\end{equation}
\end{proof}

We note that the regularity hypothesis on~$A$ is essential for
Prop.~\ref{prop:bb-diag}. As a counter-example, assume that the Jordan
sequence is~$n_1 > n_2$ with~$n_2 ≥ 2$, and let~$A = \mat{H_{n_1} &
J_{n_1,n_2} \\ J_{n_2,n_1} & H_{n_2}} = \mat{H&1\\1&H}$.
Then, using the notations from~\eqref{eq:phi-psi}, no matrix of the
form~$ψ(X) = \mat{x_1&H x_2\\x_3&x_4}$ diagonalizes~$φ(A) =
\mat{H&H\\H&H^2}$ in~$R^{2×2}$, and therefore $A$~is not not big-block
diagonalizable by a matrix commuting with~$M$.
%>>>
\begin{prop}\label{prop:diag}%<<<
Let~$A$ be an invertible, self-adjoint element of~$\ro C(M)$. Then $A$~is
congruent to a diagonal matrix, where each diagonal big block is either
the identity matrix, or the diagonal matrix~$(1, …, 1, δ)$.
\end{prop}

\begin{proof}
Use propositions~\ref{prop:bb-diag} and~\ref{prop:local-diag}.
\end{proof}%>>>


\begin{thm}\label{thm:IP1S}
Let~$k$ be a finite field of characteristic~$≠2$ and~$(b_{λ})$ be a
pencil of $n$-dimensional symmetric bilinear forms over~$k$.
It is possible, using no more than~$O(n^3)$ operations in~$k$, to compute
an isomorphism between~$(b_{λ})$ and a (unique) block-diagonal pencil
with diagonal blocks of the following form:
\begin{itemize}
\item Kronecker blocks~$K_{d} = \mat{0&K'_{d}\\\transpose{K'_d}&0}$
for integers~$d ≥ 0$;
\item local blocks~$L_{p,d,u}$, where $p$~is an irreducible polynomial,
$d ≥ 1$, and $u$~is either~$1$ or a fixed non-square element
of~$k[μ]/p(μ)$, of the form
\[ L_{p,d,u} \;= \; \mat{ & & &uλ\\ & &λ&μ\\ &⋰&⋰& \\ λ&μ& & \\}_{(d×d)}^{♭}
\;=\;
\mat{ & & &uλ\,\Id_e\\ & &λ\,\Id_e&M_p\\ &⋰&⋰& \\ λ\,\Id_e&M_p& & \\}. \]
\end{itemize}
\end{thm}

The only place where the finiteness of~$k$ is required for this theorem
to hold is for the computation of square roots.

\commentaire{Algorithme + complexité ?}

%>>>1
\section{IP2S in the regular case}%<<<1
\label{S:IP2S-regular}
Two families of polynomials~$(a_1,…,a_m)$ and~$(b_1,…,b_m)$ are
\emph{isomorphic with two secrets} if there exist bijective linear
transformations~$s$ of the $n$~variables and~$h$ of the $m$~polynomials
such that $h ∘ \bm{a} ∘ s = \bm{b}$.

Assume that $m = 2$. Then the second secret~$h$ is a homography in two
variables. Assume moreover that $a_{λ} = λ a_∞ + a_0$ and~$b_{λ} = λ
b_{∞} + b_0$ are regular pencils of quadratic forms and that $2≠ 0$
in~$k$. Then the homography~$h$ maps the characteristic polynomial~$f(λ)
= \det (a_{λ})$ to~$g(λ) = \det (b_{λ})$. In particular, it maps the
prime factors of~$f$ to those of~$g$, respecting both their degree and
their exponent as a factor of the characteristic polynomial.

Let the factors of~$f$ and~$g$ be grouped as sets~$S_{d,e}$ and~$T_{d,e}$
of factors of degree~$d$ and exponent~$e$. Then any homography~$h$
mapping all the elements of~$S_{d,e}$ to~$T_{d,e}$ for each pair~$(d,e)$
is a possible second secret in the IP2S problem. We compute the
intersection for~$(d,e)$ of the set~$H_{d,e}$ of homographies mapping the
prime polynomials of~$S_{d,e}$ to~$T_{d,e}$. In most cases, the first
set~$H_{d,e}$ already contains only one candidate, which is therefore the
second secret~$h$. The discussion depends on the degree~$d$ of the
polynomials. We note that the sum of the size of the sets~$S_{d,e}$ is
the number of variables~$n$; therefore, we may use the worst-case
estimate~$\card{S_{d,e}} = O(n)$ for each~$(d,e)$.

We shall use the following classic results.
\begin{prop}\label{prop:homography}
\begin{enumerate}
\item Let~$(x_1, x_2, x_3)$ and~$(y_1, y_2, y_3)$ be two (ordered)
triples of distinct points of~$ℙ^1(k)$. There exists a unique
homography~$h ∈ \mathrm{PGL}_2(k)$ such that~$h(x_i) = y_i$.
\item Let~$(x_1, x_2, x_3, x_4)$ and~$(y_1, y_2, y_3, y_4)$ be two
(ordered) quadruplets of distinct points. They are homographic iff they
have the same cross-ratio~$B(x) = B(y)$, where
\begin{equation}
B(x) = \frac{(x_1-x_3)(x_2-x_4)}{(x_1-x_4)(x_2-x_3)}.
\end{equation}
\item Let~$\acco{x_1, x_2, x_3, x_4}$ and~$\acco{y_1, y_2, y_3, y_4}$ be
two (unordered) sets of four points. They are homographic iff they have
the same $j$-invariant~$j(x) = j(y)$, where
\begin{equation}\label{eq:j-invariant}
j(x) = \frac{(B(x)^2-B(x)+1)^3}{B(x)^2(1-B(x))^2}.
\end{equation}
\item Let~$u(x) = ∑ u_i x^i$ and~$v(x)$ be two monic polynomials
of degree four. They are homographic iff they have the same $j$-invariant,
where $j(u)$~is a polynomial of degree~$6$ in the coefficients of~$u$.
\end{enumerate}
\end{prop}

We note that the formula for the $j$-invariant given
in~\eqref{eq:j-invariant} is, up to a constant factor, the formula for
the $j$-invariant of an elliptic curve. Namely, two elliptic curves with
equations~$y^2 = f(x)$ and~$y^2 = g(x)$, where $f, g$ are separable
polynomials of degree~$≤ 4$, are isomorphic iff the polynomials~$f$
and~$g$ are homographic.

\bigbreak
We now explain how we compute the set~$H_{d,e}$ for each pair~$(d,e)$.

\paragraph{Case~$d = 1$.}
If $\card{S_{1,e}} ≥ 3$, then we may immediately recover the
homography~$h$: namely, fix a triple~$(x_1,x_2,x_3)$ in~$S_{1,e}$, and
iterate over the triples in~$T_{1,e}$. For each such triple, there exists
a unique homography~$h$ such that~$h(x_i) = y_i$. This homography belongs
to~$H_{1,e}$ iff the images of all the other points of~$S_{1,e}$ belong
to~$T_{1,e}$. Since there are~$3!\binom{\card{S_{1,e}}}{3} = O(n^3)$
triples~$(y_i)$, this computation requires~$O(n^3)$ field operations.

If $1 ≤ \card{S_{1,e}} ≤ 2$, then $H_{1,e}$~may be explicitly computed as
the union of the set of homographies mapping the elements of~$S_{1,e}$ to
those of~$T_{1,e}$ for all permutations of~$T_{1,e}$.

\paragraph{Case~$d = 2$.}
Assume $\card{S_{2,e}} ≥ 2$. Let~$u_1, u_2 ∈ S_{2,e}$ and~$v_1, v_2 ∈
T_{2,e}$ be monic polynomials of degree~two. Any homography between the
sets~$\acco{u_1, u_2}$ and~$\acco{v_1, v_2}$ will map~$u_1 u_2$ to~$v_1
v_2$. By Prop.~\ref{prop:homography}(iv), there exists at most a bounded
number of such homographies. Since there are~$\binom{\card{S_{2,e}}}{2} =
O(n^2)$ pairs~$(v_1, v_2)$, this requires~$O(n^2)$ field operations.

If~$\card{S_{2,e}} = 1$, then $H_{2,e}$~is the set of all homographies
mapping the unique element of~$S_{2,e}$ to the unique element
of~$T_{2,e}$.

\paragraph{Case~$d = 3$.}
Fix an element~$u ∈ S_{3,e}$. For all~$v ∈ T_{3,e}$, there exist at
most~$3! = 6$ homographies~$h$ mapping~$u$ to~$v$. Each candidate belongs
to~$H_{3,e}$ iff it maps all other elements of~$S_{3,e}$ to elements
of~$T_{3,e}$. There are~$\card{S_{3,e}} = O(n)$ candidates~$u$ and
therefore~$O(n)$ candidate homographies~$h$.

\paragraph{Case~$d = 4$.}
Fix an element~$u ∈ S_{4,e}$. The candidates as homographic images of~$u$
in~$T_{4,e}$ are the~$v$ such that~$j(v) = j(u)$. Each candidate
polynomial~$v$ gives at most $4! = 24$~candidates homographies~$h$. This
allows to compute~$H_{4,e}$ in~$O(n)$ field operations.

\paragraph{Case~$d ≥ 5$.} The naïve method is to derive~$(d-4)$ times the
elements of~$S_{d,e}$ to reduce to the case where~$d = 4$. However, as
this uses only the five leading coefficients, if the polynomials are
specially chosen we may find too many homographies; for example, although
the polynomials~$x^d-1$ and~$x^d$ are not homographic, all their
derivatives are. Instead, we first compose all the elements of~$S_{d,e}$
and~$T_{d,e}$ by a known, randomly chosen homography~$r$. In general, for
any two non-homographic elements~$u_1, u_2 ∈ S_{d,e}$, the
derivatives~$(r ∘ u_i)^{(d-4)}$ are non-homographic. In the improbable
case where they are homographic, we only need to change the random
homography~$r$. In this way, we may compute the set~$H_{d,e}$ in at
most~$O(n)$ field operations.

\paragraph{Computing the hidden homography.}

The hidden homography~$h$ lies in the intersection of all sets~$H_{d,e}$.
As each one of these sets is likely to be extremely small or even reduced
to~$\acco{h}$, we compute them in increasing order of assumed complexity.
We use the above estimates: for each~$(d,e)$, we use the assumed complexity
\begin{equation}
C_{d,e} = \begin{cases}
\card{S_{d,e}}^3,& d = 1;\\
\card{S_{d,e}}^2,& d = 2;\\
\card{S_{d,e}},& d≥ 3,
\end{cases}
\end{equation}
and sort the pairs~$(d,e)$ by increasing values of~$C_{d,e}$. We finally
find a bounded number of candidate homographies using no more
than~$O(n^3)$ operations in~$k$.

% Conclusion
\section*{Conclusion}
In this paper, we show that we can solve in polynomial-time the IP problem with two quadratic forms in a finite field 
of odd characteristic. The case of even characteristic is also very important in cryptography, but it seems more difficult
to solve it. The more general problem with arbitrary number of equations $m ≥3$ is also interesting to study. 


\commentaire{reste à faire : $p = 2$ mais probablement pénible}

\commentaire{reste à faire aussi $m ≥ 3$}

% Biblio<<<1
\bibliographystyle{plain}
\bibliography{biblio}
\end{document}%<<<1

% \section{In characteristic two}%<<<1
% \subsection{Action of the commutant on diagonal matrices}
% 
% Let~$u$ be an integer. We define a $k-$linear map~$λ_u$ from $u ×
% u$-matrices to~$R_u = k[H]/H^u$ in the following way:
% \begin{equation}
% λ_u((a_{i,j})_{i,j=1,…,u}) = ∑_{i=1}^{u} a_{i,i} H^{i-1}.
% \end{equation}
% 
% \begin{prop}
% x_i^2 H^i$. Then for all~$x ∈ k[H_u]$,
% \begin{equation}
% λ_u (\transpose{x} \, A\, x) = φ(x)\, λ_u(A).
% \end{equation}
% \end{prop}
% 
% 
% \begin{proof}
% Given that $λ_u$~is $k$-linear, it is enough to prove that
% $λ_u(\transpose{H_u} \, A\, H_u) = H λ_u(A)$.
% \end{proof}
% 
% 
% \begin{lem}
% Let~$u, v$ be two integers and~$A ∈ R^{u × u}$.
% \begin{equation}
% λ_v ( \transpose{J_{u,v}}\, A\, J_{u,v}) \;=\;
% H^{\max (v-u, 0)}\,λ_u(A).
% \end{equation}
% \end{lem}
% 
% 
% \begin{prop}
% Let~$n_1 ≥ … ≥ n_r$. For any symmetric matrix~$A$ written as
% blocks~$A_{i,j}$ of size~$n_i × n_j$, define~$Λ(A)$ as the line matrix of
% length~$r$ with coefficients in~$R_1 = k[H]/H^{n_1}$:
% \begin{equation*}
% Λ(A) = ( H^{n_1 - n_i}\, λ_{n_i} (A_{i,i}))_{i=1,…, r}.
% \end{equation*}
% Then, for any~$X ∈ \ro C(M)$:
% \begin{equation}
% Λ(\transpose{X}\,A\,X) = Λ(A)\, φ(Ψ(X)).
% \end{equation}
% \end{prop}
% 
% 
% \begin{prop}
% Let~$A = \mat{Δ_1 & TM \\ 0 & Δ_2}$ and~$X = \mat{X_{1,1} & X_{1,2}\\
% X_{2,1} & X_{2,2}}$, where~$X_i ∈ \ro C(M)$ and~$Δ_i$ are diagonal.
% 
% Then $\transpose{X} A X$ has diagonal
% \[ Λ(\transpose{X} A X) = ( Λ(Δ_1) φΨ(X{1,i}) + Λ(Δ_2) φψ(X_{2,i}) +
%   Λ(\transpose{X_{1,i}} TM X_{2,i}))_{i=1,2}. \]
% \end{prop}
% 
\section{Algorithms and Complexity}
Voir avec Jerome ce qu'on met ici en fonction des connaissances ou si on le met a la fin de chaque section


\end{document}
% vim: fdm=marker fmr=<<<,>>>:

