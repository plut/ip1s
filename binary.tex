\documentclass{article}
\usepackage[utf8]{inputenc}
\usepackage[T1]{fontenc}
\usepackage{unicode}
\usepackage{math}
\usepackage[margin=25mm]{geometry}
\usepackage{color}
\DeclareUnicodeCharacter{26A0}{{\color{red}\ensuremath{\lower .25ex\hbox{\Large
$\triangle$\hskip -1.25ex}!\;\,}}}


\def\transpose#1{{\vphantom{#1}}^{\mathrm{t}}\!#1}
\def\mat#1{\begin{pmatrix}#1\end{pmatrix}}
\def\F{\mathbb{F}}
\def\appli#1#2#3#4{\begin{array}{rcl}#1&\longrightarrow&#2\\
  #3&\longmapsto&#4\end{array}}
\DeclareMathOperator\GL{GL}

\begin{document}
\title{Solving IP1S in characteristic two: the fully irregular case}

A pencil is \emph{fully irregular} if its regular part is zero,
\emph{i.e.} if it is isomorphic to a direct sum of Kronecker modules.

\section{The equations for IP1S}

Let~$k$ be a finite field with characteristic~$2$. Except where
indicated, all tensor products shall be taken over~$k$.

\subsection{Kronecker modules as homogeneous polynomials}


For~$d ≥ 0$, we write~$H_d$ for the $d+1$-dimensional space of
homogeneous polynomials in~$k[x:y]$ of degree~$d$. For any~$h ∈ H_m$, the
multiplication by~$m$ defines linear maps~$H_d → H_{d+1}$, which we again
write~$h$; all these maps
commute with each other. We write~$H_d^{∨}$ for the dual vector space
of~$H_d$, and~$h^{∨}: H_{d+1}^{∨} → H_{d}^{∨}$ for the transposed
maps of~$h$. We also write~$\chev{φ, f}$ for the canonical bilinear
map~$H_d ⊗ H_d^{∨} → k$; we note that, for~$f ∈ H_{d-1}$ and~$φ ∈
H_d^{∨}$, we always have the relation~$\chev{φ, fh} = \chev{h^{∨} φ, f}$.

The \emph{Kronecker module} of degree~$d ≥ 0$ over~$k$ is the
$2d+1$-dimensional space~$K_d = H_{d-1} ⊕ H_d^{∨}$, equipped with the
symmetric bilinear pencil~$(b_{t})$ defined for~$f, f' ∈ H_{d-1}$ and~$φ,
φ' ∈ H_{d}^{∨}$ by
\begin{equation}
b_t (f, f') \;=\; b_t (φ, φ') \;=\; 0; \qquad
b_t (f, φ) \;=\; \chev {φ, (x-ty) f}.
\end{equation}
We shall write~$u · v$ instead of~$b_t (u, v)$.

% For~$d ≥ -1$, we write~$P_d$ for the
% $d+1$-dimensional space of polynomials in~$k[x]$ of degree~$≤ d$,
% and~$P_d^∨$ for its dual. Let~$ι_d: P_{d-1} → P_{d}$ be the natural
% injection map and~$x_d: P_{d-1} → P_{d}$ be the multiplication by~$x$.
% The various maps~$ι_d$ and~$x_d$, when composable, commute with each
% other, and we shall simply write them as~$ι$ and~$x$. They define
% transposed maps~$ι^{∨}$ and~$x^{∨}: P_{

% The \emph{Kronecker module} of degree~$d$
% over~$k$ is then the space~$K_d = P_{d-1} ⊕ P_d^{∨}$, equipped with the
% symmetric bilinear pencil~$(b_{t})$ defined for~$f, f' ∈ P_{d-1}$ and~$φ, φ' ∈
% P_{d}^{∨}$ by
% \begin{equation}
% b_t (f, f') \;=\; b_t (φ, φ') \;=\; 0, \qquad
% b_t (f, φ') \;=\; \chev{φ, (x - t) ι_d f}.
% \end{equation}

\subsection{Kronecker modules with coefficients}

A \emph{coefficient space} is a bilinear space~$E$ isomorphic to~$k^n$,
together with its standard scalar product~$u, v ↦ u · v$. For any linear
maps~$α: V → E$, $β: W → E$, we write~$α · β$ for the corresponding
bilinear form on~$V × W$.

The bilinear module~$K_d^{n}$ is isomorphic to~$E ⊗ K_d$. For~$u ∈ E ⊗
H_m$ written as~$u = ∑ u_i ⊗ x^i$ with~$u_i ∈ k^n$ and~$x^i ∈ H_m$, we
define, for~$f ∈ H_d$ and~$φ ∈ H_d^{∨}$,
\begin{equation}
u f \;=\; ∑ u_i ⊗ (x^i f) \;∈ E ⊗ H_{d+m}, \quad
u^{∨} φ \;=\; ∑ u_i ⊗ ((x^i)^{∨} φ) \;∈ E ⊗ H_{d-m}^{∨}.
\end{equation}
Likewise, for any~$h ∈ \Hom(E, E') ⊗ H_m$, write~$h = ∑ h_i ⊗ x^i$
with~$h_i ∈ \Hom (E, E')$. For any~$u ⊗ f ∈ E ⊗ H_d$, we define~$h(u ⊗
f) ∈ E' ⊗ H_{d+m}$ by~$h(u ⊗ f) = ∑ h_i(u) ⊗ (x^i f)$. An element
of~$\Hom (E ⊗ H_d, E' ⊗ H_{d+m})$ defined in this way is called
\emph{principal}. In the same way, we define principal elements of~$\Hom
(E ⊗ H_{d+m}^{∨}, E' ⊗ H_{d}^{∨})$.

Let~$r ≤ s$ be two integers and~$E, E'$ be two coefficient spaces. We say
that a bilinear form~$B$ on~$(E ⊗ H_r) × (E' ⊗ H_s^{∨})$ is
\emph{principal} if it is of the form~$B(f, φ) = \chev {φ, hf} = \chev
{h^{∨} φ, f}$ for some~$h ∈ \Hom (E, E') ⊗ H_{s-r}$.

\begin{lem}\label{lem:principal}
Let~$r ≤ s$ be two integers and~$E, E'$ be two coefficient spaces.
\begin{enumerate}
\item A bilinear form~$B$ on~$(E ⊗ H_r) × (E' ⊗ H_s^{∨})$ is principal
if, and only if, for any polynomial~$z$, $B(zf, φ) = B(f, z^{∨} φ)$.
\item Let~$α: E ⊗ H_r → E' ⊗ H_s$ and~$β: E' ⊗ H_s^{∨} → E ⊗ H_r^{∨}$ be
two linear maps. If $1 · α + β · 1$~is principal, then both~$α$ and~$β$
are principal.
\end{enumerate}
\end{lem}

\subsection{The automorphisms of a direct sum of Kronecker modules}

Any totally irregular pencil of quadrics is isomorphic to a space of the
form~$⨁ E_d ⊗ K_d$, where~$E_d = k^{n_d}$, and $(n_d)$~is some finite
sequence of integers.


\begin{prop}\label{prop:aut-Ed-Kd}
The automorphisms of~$⨁ E_d ⊗ K_d$ are of the form
\[ u ⊗ f ↦ ∑_{d' ≥ d} α_{d,d'} (u) f + ∑_{d'} f^{∨} γ_{d,d'} (u), \quad
u ⊗ φ ↦  ∑_{d' ≤ d} β_{d,d'} (u)^{∨} φ, \]
where~$α_{d,d'} ∈ \Hom (E_d, E_{d'}) ⊗ H_{d'-d}$, $β_{d,d'} ∈ \Hom (E_d,
E_{d'} ⊗ H_{d-d'})$ and~$γ_{d,d'} ∈ \Hom (E_{d}, E_{d'}) ⊗
H_{d+d'-1}^{∨}$ satisfy the following relations: let~$A, B, C$ be the
matrices~$(α_{d,d'})$, $(β_{d,d'})$, and~$(γ_{d,d'})$; then
$\transpose{A} · B = 1$, and $\transpose{A} C$~is symmetric.
\end{prop}


% \begin{prop}\label{prop:aut-Ed-Kd}
% The group of automorphisms of~$⨁ E_d ⊗ K_d$ is generated by the following
% elements:
% \begin{enumerate}
% \item elements of the form~
% \begin{equation}
% u ⊗ f ↦ ∑_{d' ≥ d} α_{d, d'} (u) f, \quad
% u ⊗ φ ↦ ∑_{d' ≤ d} β_{d, d'} (u)^{∨} φ,
% \end{equation}
% where~$α_{d, d'}: E_{d} → E_{d'} ⊗ H_{d'-d}$ and~$β_{d,d'}: E_{d} → E_{d'} ⊗ H_{d-d'}$ are linear maps satisfying
% the relations, for~$d ≤ d'$
% \begin{equation}
% ∑_{i=d}^{d'} α_{d,i} · β_{d',i} =
% \begin{cases} 1 & \text{if $d = d'$,}\\0 & \text{else.}\end{cases}
% \end{equation}

% \item elements of the form~$u ⊗ φ ↦ u ⊗ φ$, $u ⊗ f ↦ ∑_{d'} γ_{d,d'} (u)
% f$, where~$γ_{d,d'}: E_d → E_{d'} ⊗ P_{d+d'}^{∨}$ is a linear map, such
% that~$γ_{d,d'} = \transpose{γ_{d', d}}$.
% \item elements of the form
% \begin{equation}
% u ⊗ φ ↦ u ⊗ φ, u ⊗ f ↦ u ⊗ f + ∑_{d'} f^{∨} γ_{d,d'} (u),
% \end{equation}
% where~$γ_{d,d'}: E_{d} → E_{d'} ⊗ H_{d+d'-1}^{∨}$ is such
% that~$\transpose{γ_{d,d'}} = γ_{d',d}$.
% \end{enumerate}
% \end{prop}

In the above theorem, we understand the elements~$u ⊗ f$ and~$u ⊗ φ$ to
belong to the spaces~$E_d ⊗ H_{d-1}$ and~$E_d ⊗ H_d^{∨}$. For~$d > d'$,
the space~$H_{d' - d}$ is zero, and therefore~$α_{d,d'} = β_{d',d} = 0$.

The relation~$\transpose{A} · B = 1$ means that, for all~$d, d'$, $∑_i
α_{d,i} · β_{d',i} = 1$ if $d = d'$ and~$0$ else; this product is to be
interpreted as the collection of the corresponding terms of all degrees
in~$(x:y)$.

The equation on~$∑
α_{d,i} · β_{d',i}$ is to be interpreted as the collection of the
corresponding terms for all degreees in~$(x:y)$. Writing the matrices~$A
= (α_{d,d'})$ and~$B = (β_{d,d'})$ ($A$~is lower triangular while $B$~is
upper triangular), these relations mean that the scalar
product~$\transpose{A} · B$ is the identity matrix. Therefore, the
elements~$α_{d,d'}$ uniquely determine all of the~$β_{d,d'}$.


\begin{proof}[Proof of Prop.~\ref{prop:aut-Ed-Kd}]
Let~$F$ be an automorphism of~$V = ⨁ E_d ⊗ K_d$. The space~$V$ is
filtered by the degrees of its kernel vectors: for all~$d$, the
sub-space~$⨁ E_{d'} ⊗ H_{d'}^{∨}$ for~$d' ≤ d$ is exactly the linear span
of all kernel vectors of degree~$≤ d$. Therefore, $F$~is upper triangular
on these spaces, so that we may write, for~$u ∈ E_d$, $φ ∈ H_{d}^{∨}$
and~$f ∈ H_{d-1}$:
\begin{equation}
F(u ⊗ f) = ∑_{d'} α_{d,d'} (u ⊗ f) + ∑_{d'} γ_{d,d'} (u ⊗ f), \quad
F(u ⊗ φ) = ∑_{d' ≤ d} β_{d,d'} (u ⊗ φ),
\end{equation}
where~$α_{d,d'}: E_d ⊗ H_{d-1} → E_{d'} ⊗ H_{d'-1}$, $γ_{d,d'}: E_d ⊗
H_{d-1} → E_{d'} ⊗ H_{d'}^{∨}$ and~$β_{d,d'}: E_d ⊗ H_{d}^{∨} → E_d ⊗
H_{d'}^{∨}$. Moreover, for all~$d$, since $β_{d,d}(u ⊗ φ)$~preserves the
kernel of~$b$, it is principal, i.e. of the form~$β_{d,d,0} (u) ⊗ φ$
where $β_{d,d,0}: E_d → E_d$~is a linear map.


Since the collection of all endomorphisms~$β$ is upper triangular, the
collection of all endomorphisms~$α$, which is its inverse transpose, is
lower triangular. We may then use Lemma~\ref{lem:principal} to prove that
all the~$α_{d,d'}$ and~$β_{d,d'}$ are principal, by induction on~$d-d'$.
% Write~$β · α$ for the bilinear product of~$β(v ⊗ φ)$ and~$α(u ⊗ f)$, so
% that we have for all~$r, s$ the relation $∑_{i ≤ s} β_{s,i} · α_{r,i}
% = 1$ if $r = s$ and $0$~else.
% We now prove by induction on~$d$ the following statement: $β_{s,s} ·
% α_{s,s} = 1$ and, for all~$r > s$, we have~$α_{r,s} = 0$. For~$s = 0$,
% the previous relation becomes~$β_{0,0} α_{r,0} = 1$ if~$r = 0$ and
% $0$~else. The case~$r = 0$ proves that~$β_{0,0}$~is invertible and
% therefore that all~$α_{r,0}$ for~$r ≥ 1$ are zero. The induction step
% works in the same way: the relation for~$s$ is now~$∑
\end{proof}



\section{Old version}

We write~$K_d$ for the Kronecker module of dimension~$2d+1$: its basis
is~$(e_0, …, e_d, f_1, …, f_d)$, with the bilinear pencil~$b_{λ}$ defined
by~$-b_{∞} (e_j, f_{j+1}) = b_{0} (e_j, f_j) = 1$ and~$0$ else. We shall
write~$e_{d,i}$ and~$f_{d,i}$ when needed to prevent ambiguity. We also
write~$K_d^+$ for the space generated by the vectors~$e_i$ and~$K_d^-$
for that generated by the~$f_i$.

Let~$k^n$ be the $n$-dimensional vector space, equipped with its
canonical scalar product and its canonical basis~$(ε_1, …, ε_n)$. Given a
sequence of integers~$(n_d)$, we write~$E_d = k^{n_d}$. The
bilinear module~$K_d^{n_d}$ is isomorphic to~$E_d ⊗ K_d$, with the basis~$ε_i ⊗
e_j, ε_i ⊗ f_j$. We order this basis in the following way: $ε_1 ⊗ e_0, …,
ε_{n_d} ⊗ e_0, …, ε_{n_d} ⊗ e_d, ε_1 ⊗ f_1, …, ε_{n_d} ⊗ f_1, …, ε_{n_d} ⊗ f_d$.

⚠ la prop. 2 est synonyme de la prop. 1, à plus bas niveau, et j'ai un
peu moins confiance dans le fait de ne pas m'être planté dans les
indices...

\begin{prop}
Let~$V = ⨁ K_d^{n_d} = ⨁ E_d ⊗ K_d$ be a module with a totally irregular
bilinear pencil. Each automorphism of~$V$ is the product, in
this order, of the following maps:
\begin{enumerate}
\item an endomorphism of the form
\[\begin{array}{rcl}
u ⊗ e_{d,i} & \longmapsto & \displaystyle ∑_{m ≥ 0} ∑_{j=0}^{m}
  α_{d,m,j} (u) e_{d-m,i-j},\\
u ⊗ f_{d,i} & \longmapsto & \displaystyle ∑_{m ≥ 0} ∑_{j=0}^{m}
  β_{d,m,j} (u) f_{d+m,i+j},
\end{array}\]
where~$α_{m,j}: E_{d} → E_{d-m}$ and~$β_{m,j}: E_{d} →
E_{d+m}$ are $k$-linear maps such that, when writing~$α_{d,m} = ∑ x^j
α_{d,m,j}$ and~$β_{d,m} = ∑ y^j β_{d,m,j}$, we have the relation
\begin{equation}
∑_{i+j=m} \transpose{α_{d+m,i}} β_{d,j} = \begin{cases}
1 & \text{if $m = 0$,}\\0 & \text{if $m ≥ 1$.}
\end{cases}
\end{equation}
\item the map defined by
\[\begin{array}{rcl}
u ⊗ e_{d,i} & \longmapsto & u ⊗ e_{d,i},\\
u ⊗ f_{d,i} & \longmapsto & u ⊗ f_{d,i} + \displaystyle
  ∑_{(d',i')} γ_{d,d',i+i'}(u) ⊗ e_{d',i'}\\
\end{array}\]
where $γ_{d,d',i}: E_d → E_{d'}$, for~$i=1,…, d+d'$, is a linear
map such that~$\transpose{γ_{d,d',i}} = γ_{d',d,i}$.
\end{enumerate}
\end{prop}

% \begin{prop}\label{prop:Aut-Kdn}
% The isometries of~$K_d^n → K_d^n$ have the matrix form
% \[ \mat{A&&&&&\\ &⋱&&&(C_{i+j})&\\&&A&&&\\
%   &&&\transpose{A^{-1}}&&\\&0&&&⋱&\\&&&&&\transpose{A^{-1}}}, \]
% where $A$~is an invertible matrix of size~$n × n$, and the
% matrices~$(C_{i})$, for~$i = 1, …, 2d$, are such that $A^{-1} C_{i}$~is
% symmetric.
% \end{prop}
% 
% \begin{proof}
% Let~$K_d[λ] = K_d ⊗_{k} k[λ]$. We recall that the kernel of~$b_{λ} =
% b_{0} + λ b_{∞}$ in~$K_d[λ]$ is the line generated by~$e(λ) = ∑ λ^j e_j$.
% 
% Let~$u: K_d^n → K_d^n$ be an isometry; it extends as an isometry of~$E ⊗
% K_d[λ]$. For all~$v ∈ k^n$, the vector~$u(v
% ⊗ e(λ)$ belongs to the kernel of~$b_{λ})$ and is therefore of the
% form~$α(v) ⊗ e(λ)$, where $α: E → E$~is a linear map. Since $u$~is an
% automorphism, the map~$α$ is invertible.
% 
% There exist linear maps~$β_{i,j}, γ_{i,j}: E → E$ such that~$u(v ⊗ f_i) =
% ∑ β_{i,j} (v) ⊗ f_j + ∑ γ_{i,j} (v) ⊗ e_j$. The relations for~$b_{λ}
% (u(v ⊗ e_i), u(v' ⊗ f_j))$ write down as
% \begin{equation}
% α(v) · β_{i,j} (v') = \text{$1$ if $r = i$, $0$ else;}
% \end{equation}
% since $α$~is regular, this means that $β_{i,i} = α' = \transpose{α}^{-1}$
% and all other~$β_{i,j}$ are zero. The relations~$b_{λ} (u(v ⊗ f_i), u(v' ⊗
% f_j)) = 0$ write down as
% \begin{equation}
% γ_{i,j}(v) · α'(v') + α'(v) · γ_{j,i} (v') = 0; \quad
% γ_{i,j-1} (v) · α'(v') + α'(v) · γ_{j,i-1} (v') = 0.
% \end{equation}
% Letting~$C_{i,j}$ be the matrix of~$γ_{i,j}$ and~$C'_{i,j} = A^{-1}
% C_{i,j}$, these relations become
% \begin{equation}
% C'_{i,j} + \transpose{C'_{j,i}} = 0, \quad
% C'_{i,j-1} + \transpose{C'_{j,i-1}} = 0.
% \end{equation}
% From these we deduce that $C'_{i,j} = C'_{i+1,j-1}$ and $C'_{i,j} +
% \transpose{C'_{i,j}} = 0$.
% \end{proof}



We write~$Δ$ for the canonical isomorphism from diagonal matrices to
column vectors, and $∇ = Δ^{-1}$. We easily check that, for any column
vector~$X$ and square matrix~$A$, we have~$Δ(A ∇(X) \transpose{A}) = σ(A)
X$, where $σ$~is the absolute Frobenius.

Let~$q$ be a quadratic pencil, with polar form
isomorphic to~$K_d^n$; then, using the above basis of~$E ⊗ K_d$, there
exist coumn vectors~$D_0, …, D_d, E_1, …, E_d$ such that $b$~has a matrix
of the form
\[ B(D_0,…, E_d) =
  \mat{∇(D_0)&&&&-λ&&\\&∇(D_1)&&&1&⋱&\\&&⋱&&&⋱&-λ\\&&&∇(D_d)&&&1\\
  &&&&∇(E_1)&&\\&0&&&&⋱&\\&&&&&&∇(E_d)}.
\]
\begin{prop}\label{prop:aut-Kdn-diag}
Let~$b$ be a quadratic pencil with matrix~$B(D_0,…, E_d)$ and~$u$ be an
isometry of~$K_d^n$, with matrix given by~$A, C_i$ as in
Prop.~\ref{prop:Aut-Kdn}. Then $b ∘ u$ has matrix $B(D'_0, …, E'_d)$,
where
\begin{eqnarray*}
D'_i &=& \transpose{σ(A)} · D_i),\\
E'_i &=& σ(A)^{-1} · E_i + ∑_j σ(\transpose{C_{i+j}}) · D_j
  - λ Δ(A^{-1}·C_{2i-1}) + Δ(A^{-1}·C_{2i}).
\end{eqnarray*}
\end{prop}

Write~$X = σ(A^{-1})$ and the symmetric matrix~$A^{-1} C_{i}$ as~$∇(Y_i)
+ Z_i + σ^{-1}(Z_i)$, where $Z_i$~is alternate.
Given two quadratic pencils with polar forms isomorphic to~$K_d^n$ and
diagonal coefficients~$D_0, …, E_d$ and~$D'_0, …, E'_d$, the IP1S problem
then reduces to the equations
\begin{equation}\begin{split}
D_i &= \transpose{X} D'_i,\\
E'_i &= X E_i + ∑ D'_j × σ(Y_{i+j}) - λ Y_{2i-1} + Y_{2i}
  + ∑ Z_{i+j} · D'_j,
\end{split}\end{equation}
where $×$~is the coefficient-wise product of the two column
vectors~$D'_j$ and~$σ(Y_{i+j})$.

\section{Solving Frobenius equations}

Part~1 shows how to bring the IP1S problem to an equation of the form
\begin{equation}
X = A σ(X) + B,
\end{equation}
where $A$~is a square matrix of size~$O(nd)$. As this matrices seem to be
quite general, we present a generic method for solving this family of
\emph{Frobenius equations}.


Let~$R$ be the non-commutative ring~$k[φ]$, with the non-commutative
relations~$φ c = σ(c) φ$ for all~$c ∈ k$. The ring $R$~has the following
properties.
\begin{itemize}
\item It is Euclidean.
More precisely, the (left-side) Euclidean algorithm works: given two
elements~$a, b ∈ k$, there exist elements~$u, v$ such that~$u a + v b =
d$ where $d$~is the gcd of~$a$ and~$b$.
\item The only two-sided ideals of~$R$ are those generated by the powers
of~$φ$.
\end{itemize}

From these two remarks and [Jacobson, Th.~19] we get the following.
\begin{thm}
Let~$φ: k^n → k^n$ be a semi-linear endomorphism. There exists a basid
of~$k^n$ in which the matrix of~$φ$ is the direct sum of cyclic matrices,
at most one of which is not nilpotent.
\end{thm}

% Write one of the cyclic parts of~$φ$ in matrix form as
% \begin{equation}
% \mat{y_0\\⋮\\y_{d-1}} = \mat{0&&&a_0\\ 1&⋱&&a_1\\ &⋱&⋱&⋮\\ &&1&a_{d-1}}
%   φ \mat{y_0\\⋮\\y_{d-1}} + \mat{b_0\\⋮\\b_{d-1}}.
% \end{equation}
Each cyclic part of~$φ$ comes down to a semi-polynomial equation in~$k$
of the form
\begin{equation}
a(σ) (y) = b,
\end{equation}
where $a$~is an element of the non-commutative ring~$R$. Let~$e = [k:
\F_2]$, so that $σ^e$~is the identity morphism. Choosing a normal basis
of~$k$ over~$\F_2$ shows that $k$~is isomorphic, as a $\F_2[φ]$-module,
to $k[φ]/(φ^e-1)$.

\end{document}
