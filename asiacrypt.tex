\documentclass{beamer}%<<<
\usepackage[utf8]{inputenc}
\usepackage[T1]{fontenc}
\usepackage{booktabs}
\usepackage{colortbl}
\usepackage{dcolumn}
\usepackage{math}
\usepackage{unicode}
\usetheme{CambridgeUS}
\useinnertheme{rectangles}

\usenavigationsymbolstemplate{}%<<<
\definecolor{bleu}{rgb}{.1,.4,.6}%1a6699
\definecolor{rouge}{rgb}{.6,.1,.15}
\setbeamercolor{structure}{fg=rouge!80}
\setbeamercolor{palette primary}{fg=white,bg=rouge!90}
\setbeamercolor{palette secondary}{fg=white,bg=rouge!80}
\setbeamerfont{frametitle}{size=\large}
\setbeamercolor{title}{fg=white,bg=bleu!60}
\setbeamercolor{title in head/foot}{fg=white,bg=bleu!90}
\setbeamercolor{section in head/foot}{fg=white,bg=bleu}
\setbeamercolor{frametitle}{fg=white,bg=bleu!80}
\setbeamercolor{block title}{fg=white,bg=bleu!80}
\setbeamercolor{block body}{fg=black,bg=bleu!20}
\setbeamercolor{author in head/foot}{parent=palette secondary}
\setbeamercolor{date in head/foot}{parent=palette primary}
\setbeamertemplate{blocks}[default]
\setbeamertemplate{title page}[default][rounded=false,shadow=false]%>>>
\makeatletter
% Playing with colortbl
\let\orig@CT@setup\CT@setup
\let\CT@tablecolor\@empty
\let\CT@tablefg\@empty
\let\CT@rowfg\@empty
\def\tablecolor#1{\def\CT@tablecolor{#1}}
\def\tablefg#1{\def\CT@rowfg{#1}}
\def\rowfg#1{\gdef\CT@rowfg{#1}}
\def\CT@setup{\orig@CT@setup
  \ifx\CT@tablecolor\@empty\else\CT@color{\CT@tablecolor}\fi
  \ifx\CT@tablefg\@empty\else\color{\CT@tablefg}\fi
  \ifx\CT@rowfg\@empty\else\color{\CT@rowfg}\fi
}
\setbeamertemplate{headline}{\leavevmode\hbox{%%<<<
  \begin{beamercolorbox}[wd=.5\paperwidth,ht=2.25ex,dp=1ex,center]
  {title in head/foot}%
    \usebeamerfont{title in head/foot}\insertshorttitle
  \end{beamercolorbox}%
  \begin{beamercolorbox}[wd=.5\paperwidth,ht=2.25ex,dp=1ex,center]
  {section in head/foot}%
    \usebeamerfont{section in head/foot}\insertsectionhead\hspace*{2ex}
  \end{beamercolorbox}%
%   \begin{beamercolorbox}[wd=.5\paperwidth,ht=2.65ex,dp=1.5ex,left]
%   {subsection in head/foot}%
%     \usebeamerfont{subsection in head/foot}\hspace*{2ex}\insertsubsectionhead
%   \end{beamercolorbox}
}\vskip0pt }%>>>
\setbeamertemplate{footline}{\leavevmode\hbox{%%<<<
  \begin{beamercolorbox}[wd=.5\paperwidth,ht=2.25ex,dp=1ex,center]
  {author in head/foot}%
    \usebeamerfont{author in head/foot}\insertshortauthor
    \expandafter\beamer@ifempty\expandafter
    {\beamer@shortinstitute}{}{~~(\insertshortinstitute)}
  \end{beamercolorbox}%
%   \begin{beamercolorbox}[wd=.5\paperwidth,ht=2.25ex,dp=1ex,center]
%   {title in head/foot}%
%     \usebeamerfont{title in head/foot}\insertshorttitle
%   \end{beamercolorbox}%
  \begin{beamercolorbox}[wd=.5\paperwidth,ht=2.25ex,dp=1ex,right]
  {date in head/foot}%
    \usebeamerfont{date in head/foot}\insertshortdate{}\hspace*{2em}
    \insertframenumber{} / \inserttotalframenumber\hspace*{2ex} 
  \end{beamercolorbox}
}\vskip 0pt }%>>>
\makeatother

% Misc
\def\transpose{{}^{\mathrm{\scriptscriptstyle t}}\!}
\def\strong#1{{\bf\color{rouge}#1}}
\def\emphz#1{\emph{{\color{bleu}#1}}}
\def\mat#1{\begin{pmatrix}#1\end{pmatrix}}
\let\mathrm\mathsf
\def\bigstrut{\leavevmode \vrule width 0pt height \baselineskip depth
.5\baselineskip}
%>>>

\begin{document}
\title[New insight into IP1S]%<<<
  {New insight into the Isomorphism of Polynomials problem IP1S and
  its use in cryptography}
\author{G. Macario-Rat\inst{1}, \underline{J. Plût}\inst{2}, H.
Gilbert\inst{3}}
\date{2013-12-02}
\institute[]{%
\inst{1}Orange Labs, \url{gilles.macario-rat@orange.fr}
% 38--40, rue du Général Leclerc, 92794 Issy-les-Moulineaux Cedex 9, France
\and
\inst{2}ANSSI, \url{jerome.plut@ssi.gouv.fr}
% 51 Boulevard de la Tour-Maubourg, 75007 Paris, France \and
\and
\inst{3}ANSSI, \url{henri.gilbert@ssi.gouv.fr}
}%
\titlegraphic{\leavevmode
% Ratio of π/4 to have the same area
\raise 1.07ex\hbox{\includegraphics[width=7.85ex]{orange-logo}}
\hfil
\includegraphics[width=10ex]{anssi-macaron}
}

\begin{frame}
\titlepage
\end{frame}%>>>
\section{Introduction}
\begin{frame}\frametitle{Isomorphism of polynomials with one secret} %<<<

We consider a field~$K$ and the algebra~$K[x_1,…,x_n]$ of polynomials in
$n$~variables.

\begin{definition}[Isomorphic polynomials]
Two families of polynomials~$(a_1, …, a_m)$ and~$(b_1, …, b_m)$
are \emphz{isomorphic} if they are related by a
bijective linear transformation~$s$ of the variables~$(x_1, …, x_n)$:
\[ a_i (x_1, …, x_n) = b_i (s_1 (x_1, …, x_n), …, s_n (x_1, …, x_n)). \]
\end{definition}

In cryptographical applications,the families~$a$ and~$b$ are public and the
transformation~$s$ is the secret (e.g. the identification protocol of
[Patarin~1996]).
\end{frame}%>>>
\begin{frame}\frametitle{The IP1S problem}%<<<

\begin{df}[Isomorphism of polynomials with one secret]
Given two families of polynomials~$(a_i)$ and~$(b_i)$:
\begin{description}[labelwidth=1ex,align=parleft]
\item[Decisional IP1S] Determine if they are isomorphic.
\item[\textbf{Computational IP1S}] If the polynomials are known in
advance to be isomorphic, compute an isomorphism~$s$.
\end{description}
\end{df}

\bigskip
Other common related problems:
\begin{itemize}
\item[MQ] Find a common root to a family of multivariate quadratic
equations (NP-complete).
\item[IP2S] Allow a linear combination of the polynomials: $t ∘ a ∘ s =
b$.
\end{itemize}
\end{frame}%>>>
\begin{frame}\frametitle{Parameters of the IP1S problem}%<<<

\hfil\begin{tabular}{cll}\toprule \color{rouge}$m$ & Number of
polynomials & (1 or 2)\\ \color{rouge}$n$ & Number of variables &
(large)\\ \color{rouge}$d$ & Degree of the polynomials & (2 or 3)\\
\color{rouge}$K$ & Base field\\\bottomrule
\end{tabular}

\bigskip
\begin{itemize}
\item The IP1S problem is easier (overdetermined) with more than 2 polynomials.
\item Key size depends on the number of polynomials and on their degree.
\item The complexity of attacks depends on the number of variables.
\end{itemize}

\bigskip
This work focuses on the case of \strong{two homogeneous quadratic
polynomials} over a finite field \strong{of any characteristic}.
\end{frame}%>>>
% \begin{frame}\frametitle{The Patarin protocol}%<<< Authentification
% protocol based on the IP1S problem. \begin{itemize} \item Public key:
% two isomorphic families of polynomials~$a$ and~$b$. \item Secret key:
% an isomorphism~$s: a → b$. \item Alice randomly generates an
% isomorphism~$t$; then~$c = a ∘ t$ is isomorphic to~$a$ (and~$b$). Alice
% sends~$c$ to~Bob. \item Bob asks either for the isomorphism between~$c$
% and~$a$, or between~$c$ and~$b$ (randomly). \item Anyone not knowing
% the secret has at most~$p={1\over2}$ of succeeding. \end{itemize}
% \end{frame}%>>>
\begin{frame}\frametitle{Previous algorithms}%<<<
\begin{itemize}
% \item{} [Levy-dit-Vehel, Perret 2004]: find algebraic equations satisfied
% by the entries of a solution matrix, and solve these equations using a
% Gröbner basis computation.
\item{} [Bouillaguet, Faugère, Fouque, Perret 2011]: transform the
problem to an overdetermined system of quadratic and linear equations.
\begin{itemize}
\item Solve experimentally the systems with Gröbner bases in time
$\widetilde O(n^6)$.
\item Solved all the quadratic IP1S challenges from [Patarin~1996]

\def\arraystretch{1.2}\tablecolor{bleu!20}
\def\w{\color{white}}\def\.{\hphantom{.}}\def\0{\hphantom{0}}
\hfil\begin{tabular}{rr}
\rowcolor{bleu!80}\w $q$ & \w $n$\\
2 & 16\\
$2^4$ & 6\\
2 & 32\\
\end{tabular}
\end{itemize}
\item This work: use structure theorems on (pairs of) quadratic forms to
reduce them to canonical forms.

\begin{itemize}
\item Uses mainly linear algebra and polynomial algebra (no Gröbner bases).
\item Requires separate treatment depending on the characteristic.
\end{itemize}
\end{itemize}
\end{frame}%>>>
% \begin{frame}\frametitle{Choosing the parameters}%<<<
% \begin{itemize}
% \item The IP1S problem is harder when there are few (1 or 2) polynomials.
% \item Attacks against the polynomials involve Gröbner basis
% computations and depend on the number~$n$ of variables.
% \item Verifying solutions requires computing the polynomials, and depends
% on the degree~$d$.
% \item The cases most often seen in cryptographical applications use small
% values of~$d$.
% \item When $d = 3$, IP1S is at least as strong as the Graph isomorphism
% problem.
% \end{itemize}
% 
% This work focuses on the case of homogeneous polynomials of degree~$d =
% 2$. In practice, many instances of this problem are solvable using
% Gröbner basis computations, but the theoretical complexity is still
% exponential.
% \end{frame}%>>>
\section{Characteristic different from two}
\begin{frame}\frametitle{Quadratic IP1S for~$m = 1$}%<<<
What about IP1S for \strong{one} polynomial?
\begin{itemize}
\item The case~$m = 1$ corresponds to isomorphism of quadratic forms of~$n$
variables.
\item To a quadratic form~$q$ we associate the \emphz{polar form}~$b$
defined by
\begin{equation*}
b(x,y) = q(x+y) - q(x) - q(y).
\end{equation*}
\item This is a symmetric bilinear form. It satisfies the \emphz{polarity
identity}
\begin{equation*}
2\,q(x) = b(x,x).
\end{equation*}
\item If $2 ≠ 0$ in~$K$, then this means that quadratic and symmetric
bilinear forms are really the same. The bilinear forms are classified by
their Gauß reduction.
\end{itemize}
\end{frame}%>>>
% \begin{frame}\frametitle{Quadratic and bilinear forms}%<<<
% \begin{itemize} \item A \emph{quadratic form} in~$x_1, …, x_n$~is a
% homogeneous polynomial~$q(x_1,…,x_n)$ of degree~two. Its \emph{polar
% form} is the symmetric bilinear form \begin{equation*} b(x,y) = q(x+y)
% - q(x) - q(y). \end{equation*} \item The polar form satisfies the
% \emph{polarity identity} \begin{equation*} 2\,q(x) = b(x, x).
% \end{equation*} \item In matrix terms: if a quadratic form has the
% matrix~$Q$, then its polar form has the matrix~$B = Q + \transpose{Q}$.
% \item Any isomorphism between (pairs of) quadratic forms gives an
% isomorphism between the associated (pairs of) polar forms. \item This
% is an equivalence if $2 ≠ 0$ in~$K$. \end{itemize} \end{frame}%>>>
\begin{frame}\frametitle{Regularity of bilinear pencils}%<<<
What about IP1S for \strong{two} polynomials?
\begin{itemize}
\item A \emphz{bilinear pencil} is an affine line in the space of bilinear
forms:
\begin{equation*}
λ ↦  b_{λ} = b_{0} + λ b_{∞}
\end{equation*}
defined by two bilinear forms $b_{∞}, b_{0}$. It is called
\begin{description}
\item[\emphz{degenerate}] if $\det b_{λ} = 0$ for all~$λ$,
\item[\emphz{regular}] if
$b_{∞}$~is regular (= invertible).
\end{description}
\item Any pencil is a direct sum
\begin{equation*}
{\text{(non-degenerate pencil)}}
\quad⊕\quad {\text{(zero pencil)}}.
\end{equation*}
\item If $(b_{∞}, b_{0})$~is not degenerate, then by
replacing~$b_{∞}$ by $b_{λ}$ where~$\det b_{λ} ≠ 0$, we may assume that
it is regular.\\(this may require a (small) extension of scalars).
\end{itemize}
\end{frame}%>>>
\begin{frame}\frametitle{Isomorphism of regular bilinear pencils}%<<<
\begin{itemize}
\item If $(b_{λ})$~is a regular pencil, then $m_{b} = b_{∞}^{-1} b_0$~is
an endomorphism of~$K^n$, which we call the \emphz{characteristic
automorphism} of~$b$. We may then write
\begin{equation*}
b_{λ} = b_{∞} (λ + m_{b}).
\end{equation*}
\item An isomorphism between the pencils~$(a_{λ})$ and~$(b_{λ})$ is a
bijective linear map~$s$ such that~$\transpose{s} · a_{λ} · s = b_{λ}$,
which is equivalent to
\begin{equation*}
\transpose{s} · a_{∞} · s = b_{∞} \qquad\text{and}\qquad
s^{-1} · m_{a} · s = m_{b}.
\end{equation*}
\item If $(a_{λ})$ and~$(b_{λ})$ are isomorphic, then $m_{a}$ and~$m_{b}$
are similar, and we may assume that they are equal.
\item The IP1S problem becomes:
\begin{equation*}
\transpose{s} · a_{∞} · s = b_{∞} \qquad\text{and}\qquad
\text{$s$~commutes with~$m$.}
\end{equation*}
where $a_{∞}$, $b_{∞}$ and~$a_0 = a_{∞} m$, $b_0 = b_{∞} m$ are symmetric.
\end{itemize}
\end{frame}%>>>
\begin{frame}\frametitle{Isomorphism of cyclic bilinear pencils}%<<<
The pencil~$(a_{λ})$~is \emphz{cyclic} if the characteristic
endomorphism~$m_{a}$~is cyclic (its characteristic polynomial is equal to
its minimal polynomial).
\begin{itemize}
\item Random instances of IP1S are generally cyclic.
\item The commuting space of~$m_{a}$~is reduced to the ring of
polynomials~$K[m_a]$.
\item The fact that $a_{∞}\, m = \transpose{m}\, a_{∞}$ means that, for
all~$s$ commuting with~$a_{∞}$, the same equation~$a_{∞}\, s =
\transpose{s}\, a_{∞}$ holds.
\item The relation~$\transpose{s}\, a_{∞}\, s = b_{∞}$ simplifies to
\begin{equation*}
a_{∞}\, s^2 = b_{∞}, \qquad \text{or}\qquad s^2 = a_{∞}^{-1}\, b_{∞},
\quad s ∈ K[m].
\end{equation*}
\item When $K$~is a finite field, this is easy to solve.
\end{itemize}
\end{frame}%>>>
\begin{frame}\frametitle{Cyclic IP1S when~$2 ≠ 0$}%<<<
\begin{theorem}[Solving cyclic IP1S in odd characteristic]
Let~$K$ be a finite field with odd characteristic and~$(a_{λ})$,
$(b_{λ})$ be two isomorphic cyclic pencils of quadratic forms of
dimension~$n$.

It is possible to compute an isomorphism between~$(a_{λ})$ and~$(b_{λ})$
using no more than~$\widetilde O(n^3)$ operations in~$K$.
\end{theorem}
\begin{itemize}
\item Computing the minimal polynomial of~$m = m_a$.
\item Computing square roots in the residual fields of~$K[m]$.
\item Lifting (Hensel) to the localizations of~$K[m]$.
\item Chinese remainders to compute the solution of~$s^2 = a_{∞}^{-1}
b_{∞}^{-1}$ in~$K[m]$.
\end{itemize}
Moreover, we know the exact number of solutions to the IP1S problem.
\end{frame}%>>>
\begin{frame}\frametitle{Computer experiments for random instances}%<<<

\def\arraystretch{1.2}\tablecolor{bleu!20}
\def\w{\color{white}}\def\.{\hphantom{.}}\def\0{\hphantom{0}}
\hfil\begin{tabular}{rrrr}
\rowcolor{bleu!80} \w $q$ & \w $n$ & \w $t$ (s) & \w \% cyclic\\
3 & 80 & 5 & 87\\
3 & 128 & 34 & 88\\
$3^{10}$ & 32 & 15 & 100\\
\end{tabular}\hfil
%
\begin{tabular}{rrrr}
\rowcolor{bleu!80} \w $q$ & \w $n$ & \w $t$ (s) & \w \% cyclic\\
5 & 20 & 0.07 & 95\\
5 & 32 & 0.28& 95\\
5 & 80 & 7\.\0\0& 95\\
\end{tabular}

\bigskip
\tablecolor{bleu!20}
\hskip 0pt plus 2fil\begin{tabular}{rrrr}
\rowcolor{bleu!80} \w $q$ & \w $n$ & \w $t$ (s) & \w \% cyclic\\
$7^6$ & 32 & 11\.\0\0 & 100\\
65537 & 8 & 0.04 & 100\\
65537 & 20 & 1\.\0\0 & 100\\
% &  &  & \\
\end{tabular}
\bigskip

\begin{itemize}
\item Opteron 850 2.2 GHz, 32 GB RAM.
\item MAGMA version 2.13-15.
\end{itemize}
\end{frame}%>>>
\section{IP1S in characteristic two}
\begin{frame}\frametitle{Quadratic forms in characteristic two}%<<<
\begin{itemize}
\item When~$2 = 0$ in~$K$, the polarity identity reads $b(x,x) = 0$,
\emph{i.e.} the polar form is an \strong{alternating} bilinear form.
\item The polarity map is not a bijection.
\item In general, a quadratic form has the decomposition
\begin{equation*}
\underbrace{\text{(regular quadratic form)}}_{\text{even dimension}}
\quad⊕\quad {\text{(sum of squares)}}.
\end{equation*}
The sum of squares is easy (semi-linear). Thus we may assume that the
polar pencil is regular.
\item We first compute all possible isomorphisms for the polar pencils,
and then look for an isomorphism that has the right action on the
diagonal coefficients.
% \item The analogue of the group of squares in~$K^{×}$ is the additive
% subgroup (of index two)~
% \begin{equation*}℘(K) = \acco {λ^2 + λ,\; λ ∈ K}.
% \end{equation*}
% {\spaceskip .31ex plus 1fil
% For example, all regular quadratic forms in dimension~$2$ are isomorphic
% to}
% \begin{equation*}
% x^2+xy+α\,y^2 \qquad \text{for $α ∈ K / ℘(K)$.}
% \end{equation*}
\end{itemize}
\end{frame}%>>>
\begin{frame}\frametitle{Pencils of alternating bilinear forms}%<<<
\begin{theorem}[Classification of alternating pencils]
Any regular pencil of alternating forms may be written, in a
suitable basis
\begin{equation*}
A_{∞} = \mat{0 & T\\T & 0}, \quad A_{0} = \mat{0 & TM\\TM & 0},
\end{equation*}
where $T$~is an invertible symmetric matrix such that $TM$~is symmetric.
\end{theorem}
\begin{itemize}
\item The endomorphism $M$~is the \emphz{Pfaffian} of~$(A_{λ})$. We may
select an appropriate representative of~$M$ in its conjugacy class (so
that for IP1S, we again have $M = M_A = M_B$), and $T$~depends only
on~$M$.
\item If the quadratic pencils~$(A_{λ})$ and~$(B_{λ})$ are isomorphic, we
may assume that both polar pencils are equal, and of the above form.
\item The pencil is called \emphz{cyclic} if $M$~is cyclic.
\end{itemize}
\end{frame}%>>>
\begin{frame}\frametitle{Automorphisms of alternating pencils}%<<<
\begin{theorem}[Structure of the orthogonal group]
The automorphisms of a cyclic pencil of alternating forms are generated
by the matrices
\begin{gather*}
G_1(x) = \mat{1 & x\\0 & 1}, \quad G_2(x) = \mat{1 & 0\\x & 1},\\
G_3(x) = \mat{x & 0\\0 & x^{-1}},\quad G_4 = \mat{0 & 1\\1 & 0},
\end{gather*}
where~$x ∈ K[M]$.
\end{theorem}

We actually have a LU decomposition: any (positive) automorphism is of the form
$G_2(y) G_3(u) G_1(x)$ for~$x, y ∈ K[M]$ and~$u ∈ K[M]^{×}$.

\end{frame}%>>>
\begin{frame}\frametitle{Normal form for alternating pencils}%<<<
\begin{itemize}
\item We may assume that the minimal polynomial~$f$ of~$M$ is of the
form~$f = f_0^{d}$, where $f_0$~is irreducible.
\item In this case, $M$~is similar to~$\tiny\mat{M_0 & 1 & & 0 \\[-2ex]
& ⋱\; & ⋱ & \\[-2ex] & & ⋱ & 1\\ 0 & & & M_0}$, where $M_0$~is the
companion matrix of~$f_0$. (This is almost the Frobenius normal form).
\item For simplicity, we present here only the case where~$M_0 = 0$. In
this case, $T$~is the anti-diagonal matrix.
\item We map diagonal matrices to~$K[M]$ in the following way:
\begin{equation*}
ψ: \quad A = \mathrm{diag}(a_0,…,a_{n-1}) ↦  α = ∑ a_i M^i ∈ K[M].
\end{equation*}
\end{itemize}
\end{frame}%>>>
\begin{frame}\frametitle{Quadratic pencils in characteristic two}%<<<
The IP1S problem reduces to: given the matrices $T$ and~$M$ as above and
diagonal matrices~$A_i$ and~$B_i$, compute an isomorphism between
\begin{equation*}\belowdisplayskip 1ex
\mat{A_1 & T\\0 & A_2}, \mat{A_3 & TM\\0 & A_4} \quad\text{and}\quad
\mat{B_1 & T\\0 & B_2}, \mat{B_3 & TM\\0 & B_4}.
\end{equation*}
We represent the diagonal matrices by elements~$α_i$ and~$β_i$ of~$K[M]$.

\begin{block}{Action of the orthogonal group on the diagonal coefficients}
\begin{itemize}
\item Let~$A$ be diagonal, $x ∈ K[M]$, and~$A'$~be the diagonal
of~$\transpose{x}\,A\,x$. Then\abovedisplayskip 1ex \belowdisplayskip 1ex
\begin{equation*}
α' = φ(x)\, α,
\end{equation*}
where $φ$~is the Frobenius map on~$K[M]$: $φ(∑ x_i M^i) = ∑ x_i^2 M^i$.
\smallskip
\item For~$x = ∑ x_i M^i ∈ K[M]$, we define
\belowdisplayskip 0pt
\begin{equation*}
θ(x) = ψ(\mathrm{diagonal}(TX)) = ∑ x_{d-1-2i} M^{d-1-i}.
\end{equation*}
\end{itemize}
\end{block}
\end{frame}%>>>
\begin{frame}\frametitle{Local equations for IP1S in $K[M]$}%<<<
Let~$s = G_2(x)\,G_3(u)\,G_1(y)$ be an orthogonal map. The action of~$s$
on the diagonal coefficients is described by \strong{four semi-linear
equations} equations on~$x$, $y$, $u$ in the algebra~$K[M]$.

\medbreak
We can eliminate~$u$ and perform a linear change of variables to reduce
IP1S to a system of the form\abovedisplayskip 1ex
\begin{equation*}\belowdisplayskip 1ex
\left\{\begin{array}{lllcl}
{\color{rouge}\alpha} \varphi(z) & +\, \hphantom{\beta} \theta(z)
  &&=&C,\\
{\color{rouge}\alpha \gamma} \varphi(x) & +\,\beta \theta(x) &+\theta(Mx)
  &=&C',\\
&\hphantom{+\,}\gamma \theta(x) + \beta \theta(z) &+\theta(Mz) &=&C''.
\end{array}\right.\end{equation*}
We note that
\begin{itemize}
\item $φ$~is bijective and preserves the valuation on~$K[M]$;
\item $θ$~is a contracting map (modulo $M^{d-1}$).
\end{itemize}

In most cases, $α γ \; (= α_1 α_4 + α_2 α_3)$~is invertible in~$K[M]$, and
using a fixed point theorem, we can solve the system in~$O(d\,\log d)$
operations in~$K$.
\end{frame}%>>>
\begin{frame}\frametitle{Solving the local equations for IP1S}%<<<
In the general case, we can study equations of the form
\begin{equation*}
M^{e} φ(x) = a\, θ(x) + b
\end{equation*}
to prove the following result:
\begin{block}{Proposition}
The local equations for IP1S may be solved using no more than~$O(d^2)$
operations in the field~$K$.
\end{block}

\end{frame}%>>>
\begin{frame}\frametitle{Solving cyclic IP1S}%<<<
\begin{theorem}[Cyclic IP1S in characteristic two]
Let~$K$ be a binary field and~$(A_{λ})$, $(B_{λ})$ be two isomorphic
cyclic pencils of quadratic forms on~$K^n$. It is possible to compute an
isomorphism from~$(A_{λ})$ to~$(B_{λ})$ using no more than~$\widetilde
O(n^3)$ operations in~$K$.
\end{theorem}
\begin{itemize}
\item Computing the characteristic polynomials.
\item Primary decomposition of~$(A_{λ})$ and~$(B_{λ})$.
\item Solving the local equations.
\item Patching via Chinese remainders to a solution of the IP1S problem.
\end{itemize}
Moreover, we can count the solutions to the IP1S problem.
\end{frame}%>>>
\begin{frame}\frametitle{Computer experiments for random instances}%<<<
\def\arraystretch{1.2}\tablecolor{bleu!20}
\def\w{\color{white}}\def\.{\hphantom{.}}\def\0{\hphantom{0}}
\hfil\hfil\begin{tabular}{rrrr}
\rowcolor{bleu!80} \w $q$ & \w $n$ & \w $t$ (s) & \w \% cyclic\\
2 & 32 & 0.07 & 96\\
2 & 128 & 2\.\0\0 & 95\\
2 & 256 & 33\.\0\0 & 94\\
$2^4$ & 32 & 0.3\0 & 100\\
$2^7$ & 32 & 0.5\0 & 100\\
\end{tabular}
\bigskip

In most cases, the determinant~$α_1 α_4 + α_2 α_3$ is invertible
in~$K[M]$, so that the quadratic convergence of the fixed point theorem
allows us to solve the local equations in~$O(d\,\log d)$.
\end{frame}%>>>
\section{Conclusion}
\begin{frame}\frametitle{Conclusion and future work}%<<<
Cyclic case:
\begin{itemize}
\item Proof of polynomiality of IP1S in all characteristics.
\item Uses classification of quadratic forms.
\item Complexity dominated by linear algebra.
\end{itemize}

\bigskip
Non-cyclic case:
\begin{itemize}
\item The commutant of the characteristic endomorphism is harder to
manipulate.
\item The IP1S problem has more solutions than in the cyclic case.
\begin{itemize}
\item For example: in the extremely non-cyclic case where $b_{0} = 0$,
the solutions are parametered by the full orthogonal group of~$b_{∞}$.
\item Giving a parametrization of the space of solutions would help
solving the problem for more than two polynomials.
\end{itemize}
\end{itemize}
\end{frame}%>>>

\end{document}
