\documentclass{beamer}%<<<
\usepackage[utf8]{inputenc}
\usepackage[T1]{fontenc}
\usepackage[francais]{babel}
\usepackage{booktabs}
\usepackage{colortbl}
\usepackage{dcolumn}
\usepackage{math}
\usepackage{unicode}
\usetheme{CambridgeUS}
\useinnertheme{rectangles}

\usenavigationsymbolstemplate{}%<<<
\definecolor{bleu}{rgb}{.1,.4,.6}%1a6699
\definecolor{rouge}{rgb}{.6,.1,.15}
\setbeamercolor{structure}{fg=rouge!80}
\setbeamercolor{palette primary}{fg=white,bg=rouge!90}
\setbeamercolor{palette secondary}{fg=white,bg=rouge!80}
\setbeamerfont{frametitle}{size=\large}
\setbeamercolor{title}{fg=white,bg=bleu!60}
\setbeamercolor{title in head/foot}{fg=white,bg=bleu!90}
\setbeamercolor{section in head/foot}{fg=white,bg=bleu}
\setbeamercolor{frametitle}{fg=white,bg=bleu!80}
\setbeamercolor{block title}{fg=white,bg=bleu!80}
\setbeamercolor{block body}{fg=black,bg=bleu!20}
\setbeamercolor{author in head/foot}{parent=palette secondary}
\setbeamercolor{date in head/foot}{parent=palette primary}
\setbeamertemplate{blocks}[default]
\setbeamertemplate{title page}[default][rounded=false,shadow=false]%>>>
\makeatletter
% Playing with colortbl
\let\orig@CT@setup\CT@setup
\let\CT@tablecolor\@empty
\let\CT@tablefg\@empty
\let\CT@rowfg\@empty
\def\tablecolor#1{\def\CT@tablecolor{#1}}
\def\tablefg#1{\def\CT@rowfg{#1}}
\def\rowfg#1{\gdef\CT@rowfg{#1}}
\def\CT@setup{\orig@CT@setup
  \ifx\CT@tablecolor\@empty\else\CT@color{\CT@tablecolor}\fi
  \ifx\CT@tablefg\@empty\else\color{\CT@tablefg}\fi
  \ifx\CT@rowfg\@empty\else\color{\CT@rowfg}\fi
}
\setbeamertemplate{headline}{\leavevmode\hbox{%%<<<
  \begin{beamercolorbox}[wd=.5\paperwidth,ht=2.25ex,dp=1ex,center]
  {title in head/foot}%
    \usebeamerfont{title in head/foot}\insertshorttitle
  \end{beamercolorbox}%
  \begin{beamercolorbox}[wd=.5\paperwidth,ht=2.25ex,dp=1ex,center]
  {section in head/foot}%
    \usebeamerfont{section in head/foot}\insertsectionhead\hspace*{2ex}
  \end{beamercolorbox}%
%   \begin{beamercolorbox}[wd=.5\paperwidth,ht=2.65ex,dp=1.5ex,left]
%   {subsection in head/foot}%
%     \usebeamerfont{subsection in head/foot}\hspace*{2ex}\insertsubsectionhead
%   \end{beamercolorbox}
}\vskip0pt }%>>>
\setbeamertemplate{footline}{\leavevmode\hbox{%%<<<
  \begin{beamercolorbox}[wd=.5\paperwidth,ht=2.25ex,dp=1ex,center]
  {author in head/foot}%
    \usebeamerfont{author in head/foot}\insertshortauthor
    \expandafter\beamer@ifempty\expandafter
    {\beamer@shortinstitute}{}{~~(\insertshortinstitute)}
  \end{beamercolorbox}%
%   \begin{beamercolorbox}[wd=.5\paperwidth,ht=2.25ex,dp=1ex,center]
%   {title in head/foot}%
%     \usebeamerfont{title in head/foot}\insertshorttitle
%   \end{beamercolorbox}%
  \begin{beamercolorbox}[wd=.5\paperwidth,ht=2.25ex,dp=1ex,right]
  {date in head/foot}%
    \usebeamerfont{date in head/foot}\insertshortdate{}\hspace*{2em}
    \insertframenumber{} / \inserttotalframenumber\hspace*{2ex} 
  \end{beamercolorbox}
}\vskip 0pt }%>>>
\makeatother

% Misc
\def\transpose{{}^{\mathrm{\scriptscriptstyle t}}\!}
\def\strong#1{{\bf\color{rouge}#1}}
\def\emphz#1{\emph{{\color{bleu}#1}}}
\def\mat#1{\begin{pmatrix}#1\end{pmatrix}}
\let\mathrm\mathsf
\def\bigstrut{\leavevmode \vrule width 0pt height \baselineskip depth
.5\baselineskip}
%>>>


\begin{document}
\section{Introduction}
\title[IP1S]%<<<
  {Avancées sur un problème d'isomorphisme de polynômes\\
  et pinceaux de formes quadratiques}
\author{G. Macario-Rat\inst{1}, \textbf{J. Plût}\inst{2}, H.
Gilbert\inst{3}}
\date{2014-01-15}
\institute[]{%
\inst{1}Orange Labs, \url{gilles.macario-rat@orange.fr}
% 38--40, rue du Général Leclerc, 92794 Issy-les-Moulineaux Cedex 9, France
\and
\inst{2}ANSSI, \url{jerome.plut@ssi.gouv.fr}
% 51 Boulevard de la Tour-Maubourg, 75007 Paris, France \and
\and
\inst{3}ANSSI, \url{henri.gilbert@ssi.gouv.fr}
}%
\titlegraphic{\leavevmode
% Ratio of π/4 to have the same area
\raise 1.07ex\hbox{\includegraphics[width=7.85ex]{orange-logo}}
\hfil
\includegraphics[width=10ex]{anssi-macaron}
}

\begin{frame}
\titlepage
\end{frame}%>>>
\begin{frame}\frametitle{Isomorphisme de polynômes à un secret}%<<<
On fixe un corps~$k$ et l'algèbre~$k[x_1,…,x_n]$ des polynômes en
$n$~variables.

\begin{definition}[Familles de polynômes isomorphes]
Deux familles de polynômes~$(a_1,…, a_m)$ et~$(b_1, …, b_m)$ sont
\emphz{isomorphes} si elles sont reliées par un changement de variables
linéaire bijectif~$s$ :
\[ a_i (x_1, …, x_n) = b_i (s_1 (x_1, …, x_n), …, s_n (x_1, …, x_n)). \]
\end{definition}

Application cryptographique : protocole d'identification de
[Patarin~1996]. Les familles $a$, $b$~sont publiques, et $s$~est le
secret. Pour prouver la connaissance de~$s$:
\begin{itemize}
\item le prouveur construit un changement de variables linéaire
aléatoire~$t$ et divulgue~$c = a ∘ t$ ;
\item le vérifieur demande aléatoirement un isomorphisme entre~$c$
et~$a$, ou entre $c$~et~$b$.
\end{itemize}
\end{frame}%>>>
\begin{frame}\frametitle{Paramètres du problème IP1S}%<<<

\hfil\hfil\begin{tabular}{cll}\toprule
\color{rouge} $m$ & Nombre de polynômes & (1 ou 2)\\
\color{rouge} $n$ & Nombre de variables & (grand) \\
\color{rouge} $d$ & Degré des polynômes & (2 ou 3) \\
\color{rouge} $k$ & Corps de base & Caractéristique ?\\
\bottomrule\end{tabular}

\bigskip
\begin{itemize}
\item Le problème IP1S est {plus facile} (surdéterminé) avec plus
de deux polynômes.
\item La taille de la clé publique dépend du nombre de polynômes et de
leur degré.
\item La complexité des attaques dépend du nombre de variables.
\end{itemize}

\bigskip
Nous considérons le cas de \strong{deux polynômes homogènes quadratiques}
sur un corps de \strong{toute caractéristique}.
\end{frame}%>>>
\begin{frame}\frametitle{Algorithmes précédents}%<<<

\begin{itemize}
\item{} [Bouillaguet, Faugère, Fouque, Perret 2011]: transforment le
problème en un système quadratique + linéaire surdéterminé.
\begin{itemize}
\item Résolvent expérimentalement (Gröbner) en temps~$\widetilde O(n^6)$.
\item Cassent toutes les tailles de paramètres proposées par
[Patarin~1996] pour le cas quadratique.
\def\arraystretch{1.2}\tablecolor{bleu!20}
\def\w{\color{white}}\def\.{\hphantom{.}}\def\0{\hphantom{0}}
\par\hfil\begin{tabular}{rr}
\rowcolor{bleu!80}\w $q$ & \w $n$\\
2 & 16\\
$2^4$ & 6\\
2 & 32\\
\end{tabular}
\end{itemize}
\item Ce travail: utilise des théorèmes de structure sur les (paires de)
formes quadratiques pour les ramener à une forme canonique.
\begin{itemize}
\item Utilise essentiellement de l'algèbre linéaire ou polynomiale (pas
de base de Gröbner).
\item Traitement spécifique pour la caractéristique~$2$.
\end{itemize}
\end{itemize}
\end{frame}%>>>
\section{Caractéristique~$≠2$}
\begin{frame}\frametitle{IP1S quadratique: le cas~$m = 1$}%<<<
Cas stupide : \strong{un seul} polynôme.

\begin{itemize}
\item Le cas~$m = 1$ correspond à calculer un isomorphisme entre deux
formes quadratiques en~$n$ variables.
\item À une forme quadratique~$q$, on associe la \emphz{forme
polaire}~$b$ définie par
\begin{equation*}
b(x,y) = q(x+y) - q(x) - q(y).
\end{equation*}
\item C'est une forme bilinéaire symétrique, reliée à~$q$ par
l'\emphz{équation de polarité} :
\begin{equation*}
2\,q(x) = b(x,x).
\end{equation*}
\item Si $2 ≠ 0$ dans~$K$, les formes quadratiques sont en bijection avec
les formes bilinéaires symétriques. Toute forme quadratique régulière est
isométrique à une forme diagonale~$(1,…,1)$ ou~$(1,…,1, δ)$, où $δ$~n'est
pas un carré.
\end{itemize}
\end{frame}%>>>
\begin{frame}\frametitle{Pinceaux de formes bilinéaires}%<<<
Cas de \strong{deux} polynômes~$(a_{∞}, a_{0})$
\begin{itemize}
\item Un \emphz{pinceau bilinéaire} est une droite projective dans
l'espace des formes bilinéaires:
\begin{equation}
λ ↦ b_{λ} = b_{0} + λ b_{∞}.
\end{equation}
Il est dit
\begin{description}
\item[\emphz{dégénéré}] si, pour tout~$λ$, $\det b_{λ} = 0$;
\item[\emphz{régulier}] si l'une des formes (par exemple~$b_{∞}$) est
régulière (= inversible).
\end{description}
\item Tout pinceau est la somme directe d'un pinceau non-dégénéré et d'un
pinceau nul.
\item On peut supposer que le pinceau non-dégénéré est régulier (quitte à
procéder à une (petite) extension des scalaires).
\end{itemize}
\end{frame}%>>>
\begin{frame}\frametitle{Isomorphisme de pinceaux réguliers bilinéaires}%<<<
\begin{itemize}
\item Soit~$b_{λ} = b_{∞} λ  + b_0 = b_{∞} ( λ + m_{b})$ un pinceau
bilinéaire régulier ; $m_{b} = b_{∞}^{-1} b_0$ est l'\emphz{endomorphisme
caractéristique} de~$b$.
\item Un isomorphisme entre~$(a_{λ})$ et~$(b_{λ})$ est une
application linéaire bijective~$s$ telle que~$\transpose{s} · a_{λ} · s =
b_{λ}$, soit encore
\begin{equation*}
\transpose{s} · a_{∞} · s = b_{∞} \qquad\text{et}\qquad
s^{-1} · m_{a} · s = m_{b}.
\end{equation*}
\item Si~$(a_{λ})$ et~$(b_{λ})$ sont isomorphes, alors quitte à effectuer
un changement de base sur~$b$, on peut supposer~$m_a = m_b$.
\item Le problème IP1S devient :
\begin{equation*}
\transpose{s} · a_{∞} · s = b_{∞} \qquad\text{et}\qquad
\text{$s$~commute avec~$m$,}
\end{equation*}
où $a_{∞}$, $b_{∞}$, $a_{∞} m$, $b_{∞} m$ sont symétriques.
\end{itemize}
\end{frame}%>>>
\begin{frame}\frametitle{Isomorphisme de pinceaux bilinéaires : le %<<<
cas cyclique}
L'endomorphisme~$m$ est \emphz{cyclique} si son polynôme minimal est égal
à son polynôme caractéristique. Dans ce cas :
\begin{itemize}
\item Le commutant de~$m$ est réduit à l'algèbre de polynômes~$k[m]$.
\item Puisque~$a_{∞}\,m = \transpose{m}\, a_{∞}$, toute~$s$
commutant à~$a_{∞}$ vérifie la même relation~$a_{∞}\,s = s\,a_{∞}$.
\item L'équation de IP1S $\transpose{s}\, a_{∞}\, s = b_{∞}$ se simplifie
en~$a_{∞}\, s^2 = b_{∞}$.
\item Il suffit de calculer une racine carrée de~$a_{∞}^{-1} b_{∞}$ dans
l'algèbre~$k[m]$, ce qui est facile si $k$~est fini.
\end{itemize}
\end{frame}%>>>
\begin{frame}\frametitle{IP1S cyclique en caractéristique impaire}%<<<
\begin{thm}[Résolution de IP1S cyclique en caractéristique impaire]
Soient~$k$ un corps fini de caractéristique impaire et~$(a_{λ})$,
$(b_{λ})$ deux pinceaux de formes quadratiques de dimension~$n$ sur~$k$,
isomorphes et cycliques.

Il est possible de calculer un isomorphisme entre~$(a_{λ})$ et~$(b_{λ})$
en~$\widetilde O(n^3)$ opérations dans~$k$.
\end{thm}
\begin{itemize}
\item Remplacer~$b$ par~$b'$ telle que~$m = m_{b} = m_{b'}$.
\item Calculer et factoriser le polynôme mimimal de~$m$.
\item Calculer les racines carrées de~$a_{∞}^{-1} b_{∞}$ dans les corps
résiduels de~$k[m]$.
\item Relever (Hensel) aux localisations de~$k[m]$.
\item Recoller (restes chinois).
\end{itemize}
De plus, on connaît le nombre de solutions.
\end{frame}%>>>
\begin{frame}\frametitle{Expérimentalement (instances aléatoires)}%<<<
\def\arraystretch{1.2}\tablecolor{bleu!20}
\def\w{\color{white}}\def\.{\hphantom{.}}\def\0{\hphantom{0}}
\hfil\begin{tabular}{rrrr}
\rowcolor{bleu!80} \w $q$ & \w $n$ & \w $t$ (s) & \w \% cyclique\\
3 & 80 & 5 & 87\\
3 & 128 & 34 & 88\\
$3^{10}$ & 32 & 15 & 100\\
\end{tabular}\hfil
%
\begin{tabular}{rrrr}
\rowcolor{bleu!80} \w $q$ & \w $n$ & \w $t$ (s) & \w \% cyclique\\
5 & 20 & 0.07 & 95\\
5 & 32 & 0.28& 95\\
5 & 80 & 7\.\0\0& 95\\
\end{tabular}

\bigskip
\tablecolor{bleu!20}
\hskip 0pt plus 2fil\begin{tabular}{rrrr}
\rowcolor{bleu!80} \w $q$ & \w $n$ & \w $t$ (s) & \w \% cyclique\\
$7^6$ & 32 & 11\.\0\0 & 100\\
65537 & 8 & 0.04 & 100\\
65537 & 20 & 1\.\0\0 & 100\\
% &  &  & \\
\end{tabular}
\bigskip

\begin{itemize}
\item Opteron 850 2.2 GHz, 32 GB RAM.
\item MAGMA version 2.13-15.
\end{itemize}
\end{frame}%>>>
\section{Caractéristique~$2$}
\begin{frame}\frametitle{Formes quadratiques en caractéristique deux}%<<<

\begin{itemize}
\item Si~$2 = 0$ dans~$k$, l'identité de polarité devient~$b(x,x) = 0$:
la forme polaire est une forme bilinéaire \strong{alternée}.
\item L'application de polarité n'est plus une bijection entre les formes
quadratiques et bilinéaires.
\item En général, on peut écrire une forme quadratique comme somme
directe
\begin{equation*}
\underbrace{\text{(forme quadratique régulière)}}_{\text{dimension
paire}} \quad ⊕ \quad \text{(somme de carrés)}.
\end{equation*}
\item La somme de carrés est facile (semi-linéaire). On travaille donc
sur les formes quadratiques régulières.
\item On décrit tous les isomorphismes entre les pinceaux polaires, et on
cherche à résoudre les équations données par leur action sur les
coefficients diagonaux.
\end{itemize}
\end{frame}%>>>
\begin{frame}\frametitle{Pinceaux de formes bilinéaires alternées}%<<<
Tout pinceau de formes bilinéaires alternées admet une base dans laquelle
il a pour matrice
\begin{equation*}
A_{∞} = \mat{0 & T\\T & 0}, \quad A_{0} = \mat{0 & TM\\TM & 0},
\end{equation*}
où $T$~est une matrice inversible telle que~$T$ et~$TM$ soient
symétriques.

\begin{itemize}
\item
L'endomorphisme~$M$ est le \emphz{Pfaffien} de~$(A_{λ})$. La matrice~$T$
ne dépend que de~$M$.
\item
Si deux pinceaux~$(A_{λ})$, $(B_{λ})$ sont isomorphes, on peut supposer
que leurs pinceaux polaires sont égaux et de la forme ci-dessus.
\item Le pinceau est dit \emphz{cyclique} si $M$~est cyclique.
\end{itemize}
\end{frame}%>>>
\begin{frame}\frametitle{Automorphismes des pinceaux alternés}%<<<
\begin{thm}[Structure du groupe orthogonal]
Le groupe d'automorphismes d'un pinceau de formes bilinéaires alternées
cyclique est engendré par les matrices
\begin{gather*}
G_1(x) = \mat{1 & x\\0 & 1}, \quad G_2(x) = \mat{1 & 0\\x & 1},\\
G_3(x) = \mat{x & 0\\0 & x^{-1}},\quad G_4 = \mat{0 & 1\\1 & 0},
\end{gather*}
pour~$x ∈ k[M]$.
\end{thm}

Plus précisément, on a une décomposition LU : tout automorphisme
(positif) est de la forme \[ G_2(y) G_3(u) G_1(x) \] pour~$x, y ∈ k[M]$
et~$u ∈ k[M]^{×}$.
\end{frame}%>>>
\begin{frame}\frametitle{Localisation du problème}%<<<
\begin{itemize}
\item Quitte à factoriser le polynôme minimal de~$M$, on suppose qu'il
est de la forme~$f_0^d$, où~$f_0$~est irréductible.
\item Dans ce cas, la matrice~$M$ est semblable à la
matrice~$\tiny\mat{M_0 & 1 & & 0 \\[-2ex]
& ⋱\; & ⋱ & \\[-2ex] & & ⋱ & 1\\ 0 & & & M_0}$, où $M_0$~est la matrice
compagnon de~$f_0$.
\item En général : extension des scalaires au corps~$k[M_0]$. Pour
simplifier : on présente uniquement le cas où~$f_0(x) = x$ (et donc
$M_0$~est la matrice~$(0)$). Dans ce cas, $T$~est la matrice
anti-identité.
\end{itemize}
\end{frame}%>>>
\begin{frame}\frametitle{Représentation algébrique de la diagonale}%<<<
Pour toute matrice~$A = (a_{i,j})$ de taille~$d × d$, on note
\begin{equation*}
ψ(A) = ∑ a_{i,i} M^{i-1} \quad ∈ R = k[M] / M^d.
\end{equation*}
La restriction de~$ψ$ aux matrices diagonales les met en bijection avec
l'algèbre~$R = k[M]$.
\begin{block}{Action du groupe orthogonal sur les coefficients diagonaux}
\begin{itemize}
\item Soient~$A$ une matrice diagonale, $x = ∑ x_i M^i ∈ k[M]$, et $A' = \transpose{x} A
x$. Alors :
\abovedisplayskip 1ex \belowdisplayskip 1ex
\begin{equation*}
ψ(A') \;=\; φ(x)\, ψ(A) \;=\; (∑ x_i^2 M^i) ψ(A).
\end{equation*}
\item Soit~$θ(x) = ψ(TX)$; alors
\begin{equation*}
θ(x) = ψ(TX) = ∑ x_{d-1-2i} M^{d-1-i}.
\end{equation*}
\end{itemize}
\end{block}
\end{frame}%>>>
\begin{frame}\frametitle{Action du groupe orthogonal sur les coefficients diagonaux}%<<<
Le problème IP1S se ramène à : étant données~$M$ et~$T$, classifier les
matrices de la forme
\begin{equation*}\belowdisplayskip 1ex
\mat{A_1 & T\\0 & A_2}, \mat{A_3 & TM\\0 & A_4}
%\quad\text{et}\quad
% \mat{B_1 & T\\0 & B_2}, \mat{B_3 & TM\\0 & B_4}.
\end{equation*}
où les~$A_i$ sont des matrices diagonales.

\begin{thm}
Soit~$α_i = ψ(A_i)$.
L'action des matrices~$G_1(x)$, $G_2(x)$, $G_3(x)$ sur les coefficients
diagonaux est donnée par :
\begin{equation*} \begin{split}
G_1^{}(x):&\quadα_{2} ← α_2 + φ(x)\, α_1\, + θ (x), \quad α_1 ← α_1;\\
G_2^{}(x):&\quadα_{1} ← α_1 + φ(x)\, α_2\, + θ (x), \quad α_2 ← α_2;\\
G_3^{}(x):&\quadα_{1} ← φ(x)\, α_1, \quad α_2 ← φ(x^{-1})\, α_2;\\
G_4^{}:&\quadα_1 \leftrightarrow α_2.
\end{split} \end{equation*}
\end{thm}
\end{frame}%>>>
\begin{frame}\frametitle{Équations locales pour IP1S}%<<<
Soit~$s = G_2(x)\,G_3(φ^{-1}(u^{-1}))\,G_1(y)$ une application
orthogonale (positive). Le problème IP1S se ramène au système de
\strong{quatre équations semi-linéaires} en~$x$, $y$, $u$ :
\begin{gather*}
u α'_1 = α_1 + α_2 φ(x) + θ(x),\\
u α_2 = α'_2 + α'_1 φ(y) + θ(y),\\
u α'_3 = α_3 + α_4 φ(x) + θ(M\,x),\\
u α_4 = α'_4 + α'_3 φ(y) + θ(M\,y).
\end{gather*}
\end{frame}%>>>
\begin{frame}\frametitle{Résolution locale de IP1S}%<<<
On peut éliminer~$u$ pour se ramener à un système en les deux
inconnues~$x$ et~$z$~:
\begin{equation*}\belowdisplayskip 1ex
\left\{\begin{array}{lllcl}
{\color{rouge}α} φ(z) & +\, \hphantom{β} θ(z)
  &&=&C,\\
{\color{rouge}α γ} φ(x) & +\,β θ(x) &+θ(Mx)
  &=&C',\\
&\hphantom{+\,}γ θ(x) + β θ(z) &+θ(Mz) &=&C''.
\end{array}\right.\end{equation*}
\begin{itemize}
\item $φ$~est bijective et conserve la valuation dans~$k[M]$; $θ$~est une
application (presque) contractante.
\item Si $α γ ( = α_1 α_4 + α_2 α_3)$~est inversible
dans~$k[M]$, le théorème du point fixe permet de résoudre le système
en~$O(d)$ étapes.
\item Dans le cas général : en étudiant des équations de la forme
$M^{e} φ(x) = a\, θ(x) + b$, on peut résoudre en~$O(d)$ étapes.
\end{itemize}
\end{frame}%>>>
\begin{frame}\frametitle{Résolution de IP1S cyclique en caractéristique deux}%<<<
\begin{thm}[IP1S cyclique en caractéristique 2]
Soient~$k$ un corps binaire et~$(A_{λ})$, $(B_{λ})$ deux pinceaux de
formes quadratiques sur~$k^n$, isomorphes et cycliques. Il est possible
de calculer un isomorphisme entre~$(A_{λ})$ et~$(B_{λ})$ en au
plus~$\widetilde O(n^3)$ opérations dans~$k$.
\end{thm}
\begin{itemize}
\item Calculer et factoriser les polynômes caractéristiques.
\item Décomposition primaire des deux pinceaux.
\item Résolution des équations locales (théorème de point fixe).
\item Assemblage (restes chinois).
\end{itemize}
De plus, il est raisonnable de compter les solutions du problème.
\end{frame}%>>>
\begin{frame}\frametitle{Expérimentalement (instances aléatoires)}%<<<
\def\arraystretch{1.2}\tablecolor{bleu!20}
\def\w{\color{white}}\def\.{\hphantom{.}}\def\0{\hphantom{0}}
\hfil\hfil\begin{tabular}{rrrr}
\rowcolor{bleu!80} \w $q$ & \w $n$ & \w $t$ (s) & \w \% cyclique\\
2 & 32 & 0.07 & 96\\
2 & 128 & 2\.\0\0 & 95\\
2 & 256 & 33\.\0\0 & 94\\
$2^4$ & 32 & 0.3\0 & 100\\
$2^7$ & 32 & 0.5\0 & 100\\
\end{tabular}
\bigskip

(Note : en général, le déterminant~$α_1 α_4 + α_2 α_3$ est inversible
dans~$k[M]$, et la résolution locale est dominée par le théorème du point
fixe).
\end{frame}%>>>
\section{Conclusion}
\begin{frame}\frametitle{Conclusion}
Dans le cas cyclique :
\begin{itemize}
\item Preuve de la polynomialité de IP1S en toute caractéristique.
\item Complexité dominée par l'algèbre linéaire et polynomiale.
\item Utilise la classification des formes quadratiques.
\end{itemize}

Restent à faire :
\begin{itemize}
\item Le cas non-cyclique \strong{(résolu en caractéristique impaire)}.
\item Le cas de~$≥ 3$ formes quadratiques.
\item Les équations cubiques...
\end{itemize}
\end{frame}



\end{document}
