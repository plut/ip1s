\documentclass{article}
\usepackage{amsmath,amsfonts,amssymb}
% \usepackage{math}
\usepackage[T1]{fontenc}
\usepackage[utf8]{inputenc}
\usepackage[margin=30mm]{geometry}
\usepackage{algpseudocode}
\usepackage{algorithm}

\newtheorem{df}{Definition}
\newtheorem{thm}{Theorem}
\newtheorem{prop}{Proposition}
\newenvironment{proof}{\textit{Proof.}~}{~$\lhd$}
\def\transpose#1{{}^{\mathrm{t}}\!#1}
\def\End{\mathrm{End}}
\def\Otilde#1{\ensuremath{\widetilde O\left(#1\right)}}
\def\labelenumi{(\roman{enumi})}

\begin{document}
Throughout this document, we fix a finite field~$k$ of characteristic~$p
\neq 2$. Let~$V$ be an $n$-dimensional $k$-vector space.

\begin{df}
Let~$Q(V)$ be the $k$-vector space of all quadratic forms on~$V$.
A \emph{projective pencil of quadratic forms} on~$V$ is the datum of a
projective line in~$\mathbb{P}Q(V)$, \emph{i.e.} a two-dimensional vector subspace
of~$Q(V)$. An \emph{affine pencil of quadratic forms} is
the datum of an affine line in~$Q(V)$, or equivalently a pair of elements
of~$Q(V)$.
\end{df}

The affine pencil with basis~$(A_{0}, A_{\infty})$ is written~$A_{0} + \lambda
A_{\infty}$. Given a projective pencil~$A$ of~$Q(V)$, the choice of any
basis~$(A_0, A_{\infty})$ of~$A$ determines the affine pencil~$A_{0} + \lambda
A_{\infty}$. (The terminology~$(A_{0}, A_{\infty})$ emerges when one considers the
pencil as a projective morphism from the projective line~$\mathbb{P}^1$ into the
space of quadrics: then $A_{0}$ and~$A_{\infty}$ are the respective images
of~$0$ and~$\infty$).

\begin{df}
A projective pencil is \emph{regular} if it contains at least one regular
quadratic form. An affine pencil~$(A_{0}, A_{\infty})$ is \emph{regular} if
the quadratic form~$A_{\infty}$ is regular; it is \emph{degenerate} if the
intersection of the kernels of all quadratic forms~$A_{\lambda} = A_0 + \lambda
A_{\infty}$ is not zero.
\end{df}

If $A$~is a regular projective pencil, then there exists a basis~$(A_{0},
A_{\infty})$ such that the corresponding affine pencil is regular. Conversely,
if~$A_{0} + \lambda A_{\infty}$ is a non-degenerate, but maybe non-regular, affine
pencil, then the corresponding projective pencil is regular and therefore
has a regular affine pencil as a representative. Concretely, this means
that there exists a scalar~$\mu$ such that the affine pencil~$A_0 + \lambda
(A_{\infty} + \mu A_0)$ is regular.

\begin{df}
Two affine pencils~$A_\lambda = A_0 + \lambda A_{\infty}$ and~$B_\lambda = B_0 + \lambda B_{\infty}$ are
\emph{congruent} if there exist an invertible endomorphism~$P \in
\End(V)$ such that, for all~$\lambda \in k$, $\transpose{P} A_{\lambda} P = B_{\lambda}$.
\end{df}

We are concerned with the following problem (Pencil congruence problem):
\emph{given two affine pencils~$A_{\lambda}$ and~$B_{\lambda}$ that we know to be
congruent, find any invertible endomorphism~$X \in \End(V)$ such
that~$\transpose{X} A_{\lambda} X = B$.} We first note that we may assume both
pencils to be regular. Namely, if both are degenerate then we may
quotient out their (common) kernel and assume that both pencils are
non-degenerate; in this case, a suitable change of basis in the projective
pencils brings us back to the case of two regular affine pencils.

\begin{prop}
Let~$A_{\lambda} = A_0 + \lambda A_{\infty}$, $B_{\lambda} = B_{0} + \lambda B_{\infty}$ be two
regular affine pencils.
\begin{enumerate}
\item If $A_{\lambda}$~is congruent to~$B_{\lambda}$, then the matrices
\[ \Omega_{A} = A_{\infty}^{-1} A_0 \quad\text{and}\quad \Omega_{B} = B_{\infty}^{-1} B_0 \]
are similar.
\item Assume that~$\Omega_{A}$ and~$\Omega_{B}$ are similar and
choose~$P$ such that~$P^{-1} \Omega_{A} P = \Omega_{B}$. Then
$\transpose{P} A_{\lambda} P =
\transpose{P} A_{\infty} P (\lambda - \Omega_{B})$.
\item Assume that~$A_{\lambda} = A_{\infty} (\lambda - \Omega)$ and $B_{\lambda} = B_{\infty} (\lambda - \Omega)$.
Then the solutions of the pencil congruence problem are exactly the
invertible~$X$ commuting with~$\Omega$ and such that~$\transpose{X} A_{\infty} X =
B_{\infty}$.
\end{enumerate}
\end{prop}

\begin{proof}
(i)
Since $A_{\lambda}$~is regular, $A_{\infty}$~is invertible and we may write $A_{\lambda} =
A_{\infty} (\lambda + A_{\infty}^{-1} A_0)$; likewise, $B_{\lambda} = B_{\infty} (\lambda + B_{\infty}^{-1}
B_0)$. Choose~$P$ such that~$\transpose{P} A_{\lambda} P = B_{\lambda}$: then
\begin{equation}
B_{\infty} (\lambda + \Omega_B) \;=\; \transpose{P} A_{\lambda} P \;=\;
\transpose{P} A_{\infty} P (\lambda + P^{-1} \Omega_A P),
\end{equation}
which implies~$P^{-1} \Omega_A P = \Omega_B$ as required. The same
computations also prove~(ii).

(iii) follows directly from the equality
$\transpose{X} A_{\infty} (\lambda - \Omega) X \;=\;
\transpose{X} A_{\infty} X (\lambda - X^{-1} \Omega X)$.
\end{proof}


\begin{prop}
Assume that~$A_{\lambda} = A_{\infty} (\lambda - \Omega)$ and~$B_{\lambda} (\lambda - \Omega)$ are both regular
and symmetric and that the matrix~$\Omega$ is cyclic, that is, its minimal and
characteristic polynomials are equal.

Then the solutions~$X$ of the pencil congruence problem are the square
roots of~$A_{\infty}^{-1} B_{\infty}$ in the algebra~$k[\Omega]$.
\end{prop}

\begin{proof}
Since $\Omega$~is cyclic, its commutant is reduced to the algebra~$k[\Omega]$;
therefore, all solutions of the congruence problem are polynomials
in~$\Omega$.

Since $A_{\lambda}$~is symmetric, both matrices~$A_{\infty}$ and~$A_{0} = A_{\infty} \Omega$
are symmetric; therefore, $\transpose{\Omega} A_{\infty} = A_{\infty} \Omega$. Since $X$~is a
polynomial in~$\Omega$, we deduce that also~$\transpose{X} A_{\infty} = A_{\infty} X$.

The relation~$\transpose{X} A_{\infty} X = B_{\infty}$ may therefore be rewritten
as~$A_{\infty} X^2 = B_{\infty}$, or~$X^2 = A_{\infty}^{-1} B_{\infty}$.
\end{proof}

We call a pencil~$A_0 + \lambda A_{\infty}$ \emph{cyclic} when the
endomorphism~$A_{\infty}^{-1} A_0$ is cyclic, and note in passing that
this is actually a property of the projective pencil.

\begin{prop}
Let~$A_{\lambda}$, $B_{\lambda}$ be two regular pencils of quadrics
over~$k^n$, congruent to each other, such that at least one is cyclic
(and therefore both are). Then the pencil congruence problem may be
solved using no more than~\Otilde{n^3} scalar operations.
\end{prop}

\begin{algorithm}
\caption{Solve the pencil congruence problem (odd characteristic, generic
case) for pencils~$A_{\lambda}$ and~$B_{\lambda}$}
\begin{algorithmic}[1]
\State\label{step:jordan}%
  Find $P$ such that~$P^{-1} A_{\infty}^{-1} A_0 P =
	B_{\infty}^{-1} B_0$.
\State\label{step:define}%
  Define $A'_{\infty} \leftarrow \transpose{P} A_{\infty} P;
	\quad \Omega \leftarrow B_{\infty}^{-1} B_0;
	\quad C \leftarrow A'_{\infty}{}^{-1} B_{\infty}.$
\State\label{step:min}%
  Compute~$f = $ minimal polynomial of~$\Omega$ and abort if $\deg f < n$.
\State\label{step:factor}%
  Factor~$f = \prod p_i^{e_i}$ over~$k$.
\State \label{step:polynomial}%
  Write~$C$ as a polynomial~$g(\Omega)$.
\ForAll{$i$}
  \State\label{step:crt1}%
	Compute~$y_i = $ image of~$C$ in~$k[t]/p_i(t)^{e_i}$.
	\State\label{step:root}%
	Compute~$\overline{x}_i = $ a square root of~$y$ in the residual
	field~$k[t]/p_i(t)$.
	\State\label{step:hensel}%
	Lift~$\overline{x}_i$ to a square root~$x_i$ of~$y$
	in~$k[t]/p_i(t)^{e_i}$.
\EndFor
\State\label{step:crt2}%
Using all~$x_i$, compute the square root~$X$ of~$C$ in~$k[\Omega]$.
\State\textbf{Return} $PX$.
\end{algorithmic}
\end{algorithm}


\begin{proof}
\textbf{Step~\ref{step:jordan}}
may be done using for example Jordan reduction of both
matrices~$A_{\infty}^{-1} A_0$ and~$B_{\infty}^{-1} B_0$, which has a complexity
of~\Otilde{n^3} scalar operations.
\textbf{Step~\ref{step:min}} requires no more than~\Otilde{n^3} scalar
operations using the Berlekamp-Massey algorithm.
\textbf{Step~\ref{step:factor}} requires at most~\Otilde{n^2} scalar
operations using Cantor-Zassenhaus.
\textbf{Step~\ref{step:polynomial}} may be done by computing the minimal
polynomial~$\mu$ of~$C$ and then finding an appropriate root~$c = g(\Omega)$
of~$\mu$ in the algebra~$k[\Omega]$; as previously, this needs at
most~\Otilde{n^3} scalar operations.
\textbf{Step~\ref{step:crt1}} is a simple Chinese remainder computation
and \textbf{Step~\ref{step:root}} is a square root computation, with
complexity~\Otilde{(\deg p_i)^3} using Cipolla's algorithm.
\textbf{Step~\ref{step:hensel}} is a Hensel lift, with
complexity~\Otilde{1} operations in~$k[t]/p_i(t)^{e_i}$.
Finally, \textbf{Step~\ref{step:crt2}} is again a Chinese remainder
computation, and the final step is a matrix multiplication.
\end{proof}

% Since we know in advance that the pencil congruence problem has a
% solution, in the cyclic case we know that the matrix~$C = A_{\infty}^{-1}
% B_{\infty}$ belongs to~$k[\Omega]$ and has a square root in this algebra. Here we
% recall how to compute this square root.
% 
% Let~$f$ be the minimal polynomial of~$\Omega$ over~$k$ and~$f = \prod p_i^{e_i}$
% be its factorization. Then~$k[\Omega] = k[\Omega]/f(\Omega)$ is isomorphic to~$\prod k[t_i]
% / p_i(t_i)^{e_i}$, the isomorphism being computable via Chinese
% remainders; thus we only need to compute the square root of the image~$y$
% of~$C$ in the local algebra~$R = k[t]/p(t)^{e}$.
% 
% Write~$\pi = p(t) \in R$: then~$\pi^e = 0$ and $\pi$~generates the maximal ideal
% of~$R$; moreover, $R/\piR$~is the finite extension~$ℓ = k[t]/p(t)$ of~$k$,
% and~$R = ℓ[\pi]/\pi^e$. Let~$v$ be the $\pi$-adic valuation of~$y$; since the
% pencil congruence problem has a solution, we know that~$v$~is even, and
% we have~$√y = \pi^{v} √{\pi^{-2v} y}$; we may therefore assume that~$v = 0$.
% Let~$x_0$ be a square root of the image of~$y$ in~$ℓ$. Using Hensel's
% lemma, we define a sequence~$(x_i)$ in the following way: $x_{i+1} =
% \frac 12 x_i + \frac{y}{2 x_i}$ and deduce that~$x_i^2 ≡ y
% \pmod{\pi^{2^i}}$.
% 
\end{document}
